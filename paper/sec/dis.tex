\section{Discussion}\label{sec:discussion}
In this work, we demonstrate that a bimodal structure in the nuclear abundance plane is associated with a brief quiescent phase. We now construct a possible framework for understanding the formation of the bimodality in our simulation, discuss its relation to prior work, and then discuss predictions and connections to observations.

\subsection{Quiescence Leads to Bimodality}\label{ssec:formqui}
We executed a series of idealized merger simulations in which we modified the starting radius and velocity by $\pm10\%$ and the circularity by $\pm0.1$, for a total of $27$ simulations. The central and satellite galaxies, which are meant to resemble the Milky Way and GSE at $z\sim2$, are otherwise identical across the simulations. Some of these simulations induce a bimodality, while others do not. We have examined a representative of each scenario in detail.

We described a series of results in Section~\ref{sec:results} which give a natural explanation for the separation between the high- and low-$\alpha$ sequences in our simulations. First, a merger induces a starburst in the simulation. If this starburst does not induce a quiescent period, then the median \MgFe{} of the gas briefly increases before continuing along the typical evolutionary path (right panel of Figure~\ref{fig:SFR_alpha}). However, if there is a quiescent period, the median \MgFe{} drops for the duration of the quiescent period before recovering to the typical path (left panel of Figure~\ref{fig:SFR_alpha}).

This drop can be understood in terms of the mechanics of the galaxy formation model as described in Section~\ref{ssec:gfm}. When a star particle forms, it only takes $\sim40\,\Myr$ before the last Type II SN has gone off and iron enrichment is dominated by Type Ia SNe. As a result, during the quiescent phase there is relatively little $\alpha$-element production. On the other hand, star particles are still able to generate iron through Type Ia SNe which have a much longer period of influence. In the TNG model, the number of Type Ia SNe decays with time as $(t/\tau_8)^{-s}$, where $\tau_8=40\,\Myr$ is the lifetime of an $8\,\Msun$ star and $s=1.12$. Therefore, the burst of star formation at $t\sim2.25\,\Gyr$ is able to generously produce iron starting at $2.5\,\Gyr$, when the drop is the most steep.

\subsection{Dilution Interpretation}\label{ssec:dilute}
In some previous work, it was reported that the bimodality is a consequence of a sudden deposition of metal-poor, $\alpha$-poor gas by a satellite or cosmological filament -- i.e., a ``dilution'' \citep{2020MNRAS.491.5435B,}. The separation of the sequences follows from the rapidity of the dilution.

We have shown that minor changes to the orbit of our idealized merger can result in outcomes that are either bimodal or unimodal. The content of gas that is delivered to the system is similar between the two regimes, and so our simulations do not support the dilution interpretation. However, this may be because the inflowing gas is too enriched, arising from an unrealistic feedback and enrichment routine, or from the unrealistic aspects of our setup. Our work does argue against the necessity of the dilution interpretation, but we also cannot completely rule it out.

\subsection{Direct Search for Quiescence}\label{ssec:obsqui}
Figure~\ref{fig:before_after_sfh_by_iron} indicates a very direct observational test of the mechanism proposed in this work: for disk stars at a given \FeH{}, there should be a gap of $\sim300\,\Myr$ in ages at $\sim8\,\Gyr$. With a survey of properly chosen stars, this could potentially be achieved with a modest (few hundred) sample of stars with age uncertainties of $\sim1\%$. The present observational landscape for age dating of old stars is far from this goal. To our knowledge, the best method at these ages is differential analysis of solar twins, which can provide an age uncertainty of $\sim5\%$ \citep[e.g.][]{2014ApJ...795...23B,2018MNRAS.474.2580S}. However, this only works for stars with solar metallicity, where there is not a clear separation between the high- and low-$\alpha$ sequences (though a gap in ages may still be present). 

And so, this test may only be possible to a reader in the distant future. However, the situation might not be completely hopeless. The necessary sample size is small, and so copious time and instruments can be dedicated. Moreover, only \textit{relative} ages need to be determined, so a sort of differential approach may yield interesting results. In addition to the presence of the gap, our work also predicts that the timing of the gap is \FeH{} dependent.

\subsection{Observational Evidence of Quiescence}\label{ssec:obshiz}
Our main argument in this paper is that the Milky Way underwent a brief quiescent phase induced by a starburst at $z\sim2$. In observations, a Milky Way-analogue would be identified as a post-starburst galaxy (PSB). With abundance matching, we expect the Milky Way's total stellar mass to be $\sim2\times10^{10}\,\Msun$. A number of authors have explored PSBs and quiescent galaxies of this mass at $z\sim2$, with large advances in the post-JWST era.

First, PSBs are not uncommon. \citet{2023ApJ...953..119P} found that in massive galaxies ($M_* > 10^{10.6}\,\Msun$) the fraction of PSBs increases from $\sim2.7\%$ ($99/3655$) at $1.0 < z < 1.44$ to $\sim8\%$ ($89/1118$) at $2.16 < z < 2.5$, confirming previous work that PSBs are more common towards $z\sim2$ \citep[e.g.,][]{2016MNRAS.463..832W}. If these galaxies can be quickly rejuvenated as the system studied in this work would suggest, then the true fraction of galaxies that go through a starburst-quenching phase may be higher.

\citet{2023arXiv231215012C} found that lower mass quiescent galaxies (towards $10^{10.3}\,\Msun$) tend to be younger and more disky, pointing to a merger driven scenario. \citet{2023arXiv231212207A} found that at even lower masses ($<10^{9.5}\,\Msun$) Furthermore, the AGN quenching mechanism, which we believe to be responsible for the quenching in our system (Appendix~\ref{app:cause_qui}), is thought to operate at these redshifts \citep[e.g.][and references therein]{2023arXiv230805795B}.

\subsection{Connection to Cosmological Simulations}\label{ssec:cosmo}
As discussed in Section~\ref{sec:intro}, several authors have examined the formation of abundance plane structure in cosmological simulations. Of most interest to us is the zoom Au~23 in \citet{2018MNRAS.474.3629G}. This galaxy, one of six considered in their work, exhibits a bimodality that extends beyond the inner disk. The interpretation given by the authors is of a ``shrinking'' gaseous disk. This is equivalent to saying that the outer disk becomes depleted of gas. This shrinking of the disk, which occurs at $t_{\textrm{lookback}}\sim6\,\Gyr$, is associated with a dip in the SFR at that radius and a decrement in the median \alphaFe{} of $\sim0.05\,\dex$ (their Figure~2), which shortly after recovers. This sequence of events is more extended than in our work, but it strongly resembles the scenario in Figure~\ref{fig:before_after_sfh}.

\citet{2018MNRAS.477.5072M} found that Milky Way-like bimodalities are rare in EAGLE, occuring in $\sim5\%$ of galaxies. \citet{2021MNRAS.501..236D,2022MNRAS.515.1430D} showed that merger-induced quenching in zooms can occur in the EAGLE model \citep[see also][]{2017MNRAS.465..547P}. However, the situation may be different in the lower resolution large box.

\citet{2023arXiv231016085K} explored the impact of quenching in the Magneticum Pathfinder suite. They found that galaxies which quench undergo a starburst followed by an AGN-driven quenching phase. In the post-starburst regime, they claim galaxies are $\alpha$-enhanced. From their Figure~8 it is difficult to quantify the difference between the quenched and non-quenched galaxies as the two populations almost completely overlap. We do find that the bulk stellar \MgFe{} is enhanced after the merger in our fiducial simulation, but only at the $\sim0.01\,\dex$ level.

\subsection{Future Work}
A natural next step would be to extend the idealized simulations in this work to a wider range of orbits, galaxy properties, and feedback models. However, the unrealism of our setup, which aids interpretation, limits its applicability to the real universe. And so something along the lines of the genetic modification technique to explore various mergers as done in this work may be useful \citep{2016MNRAS.455..974R,2017MNRAS.465..547P}.

There is, of course, still great uncertainties in the stellar evolution models commonly adopted by different groups. Initial work on systematically exploring the stellar evolution parameters has been done by \citet{2017A&A...605A..59R,2021MNRAS.508.3365B}. Exploring these variations in the simulations presented in this work would be interesting, though exploring their interactions with brief quiescent periods in simpler chemical evolution models may be a better first step.

There is also the perennial problem of diffusion within the hydrodynamics solver. In purely Lagrangian solvers, there is no diffusion between resolution elements, while in Eulerian codes the diffusion can be quite high. AREPO limits the numerical diffusion by allowing the mesh to move in a quasi-Lagrangian manner, and using a second-order solver \red{cite}. In FIRE-2 \citep{2018MNRAS.480..800H}, which uses the Lagrangian code GIZMO \citep{2015MNRAS.450...53H}, a subgrid turbulent metal diffusion model was used. It would be interesting to see how models with different diffusivity properties would relax or strengthen the necessity of a quiescent period to produce a bimodality.

% JWST quiescent galaxies https://arxiv.org/abs/2404.12432
