\section{Methods}\label{sec:methods}
\subsection{Simulation Setup}\label{ssec:setup}
Our goal in setting up the simulations in this work is not an attempt at matching any Milky Way observable in any detail. Our aim is to construct a setup resembling the merger between the Milky Way and GSE at $z\sim2$. However, there are great uncertainities associated with the state of the Milky Way and of GSE at this time. In Appendix~\ref{app:ics}, we discuss in more detail the choices made in tuning our setup to arrive at a system which is reasonably useful for understanding the Galaxy.

Here, we provide an overview of some salient details of our setup. The interested reader is referred to Appendix~\ref{app:ics} for more details. In isolation, each galaxy is a compound halo setup, with a Hernquist dark matter halo and a gaseous halo with a $\beta$-profile. The dark matter halo is initialized to be in gravitational equilibrium with the total potential. The gaseous halo is in gravito-hydrostatic equilibrium, where the temperature is allowed to vary as a function of radius. The azimuthal velocity of the gaseous halo is given as a fraction of the circular velocity. There is no initial stellar disk or bulge, and the gaseous halo is initially metal-free. All star particles and metals are formed self-consistently. Both the central and satellite galaxies are setup in this way.

In order to combine the galaxies, we follow \citet{2021ApJ...923...92N}, and place the satellite galaxy on a retrograde orbit. In the fiducial simulation, the satellite is placed at the virial radius with the virial velocity with a circularity of $0.5$. In order to test minor changes to the orbit, we ran a grid of simulations with $\pm10\%$ in each the starting radius and velocity, and $\pm0.1$ in the circularity, for a total of $27$ simulations.