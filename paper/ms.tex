\documentclass[twocolumn,linenumbers,trackchanges]{aastex631}

\usepackage{graphicx}
\usepackage{amsmath}
\usepackage{amssymb}
\usepackage{newtxtext,newtxmath}
\usepackage{hyperref}
\usepackage{gensymb}
\usepackage{enumitem}

\newcommand{\vdag}{(v)^\dagger}
\newcommand\aastex{AAS\TeX}
\newcommand\latex{La\TeX}

\newcommand{\Msun}{\ensuremath{M_{\odot}}}
\newcommand{\Gyr}{\ensuremath{\textrm{Gyr}}}
\newcommand{\Myr}{\ensuremath{\textrm{Myr}}}
\newcommand{\yr}{\ensuremath{\textrm{yr}}}
\newcommand{\kpc}{\ensuremath{\textrm{kpc}}}
\newcommand{\pc}{\ensuremath{\textrm{pc}}}
\newcommand{\kms}{\ensuremath{\textrm{km}/\textrm{s}}}
\newcommand{\tocite}{\textcolor{blue}{cite}}
\newcommand{\FeH}{\ensuremath{[\textrm{Fe}/\textrm{H}]}}
\newcommand{\MgFe}{\ensuremath{[\textrm{Mg}/\textrm{Fe}]}}
\newcommand{\alphaFe}{\ensuremath{[\alpha/\textrm{Fe}]}}
\newcommand{\tform}{\ensuremath{t_{\textrm{form}}}}
\newcommand{\dex}{\ensuremath{\textrm{dex}}}
\newcommand{\Msunyr}{\ensuremath{\Msun/\textrm{yr}}}

\newcommand{\red}[1]{\textcolor{red}{#1}}

% \received{March 1, 2021}
% \revised{April 1, 2021}
%\accepted{\today}

\shorttitle{The Milky Way's Phoenix Phase}
\shortauthors{Beane}

\graphicspath{{./}{fig/}}

\begin{document}

\title{Rising from the Ashes: A Metallicity-Dependent Star Formation Gap Splits the Milky Way's $\alpha$-Sequences}

\author{Angus Beane}
\affiliation{Center for Astrophysics $|$ Harvard \& Smithsonian, Cambridge, MA, USA}

\begin{abstract}
    The elemental abundance distribution of stars encodes the history of the gas-phase abundance in the Milky Way. Without a large, unbiased sample of highly precise stellar ages, the exact timing and nature of this history must be \textit{inferred} from the abundances. In the two-dimensional plane of \alphaFe{}-\FeH{}, it is now clear that two separate populations exist -- the low-$\alpha$ and high-$\alpha$ sequences. \edit1{We propose that a brief ($\sim300\,\Myr$) halt in star formation within a narrow metallicity bin can lead to a bimodal \alphaFe{} distribution at that metallicity, assuming a rapidly declining gas phase \alphaFe{}. Using simulations of an idealized setup of a high-$z$ galaxy merger, we show that the merger with the Gaia-Sausage-Enceladus satellite at $z\sim2$ is one possible way to trigger such a gap in the Milky Way. Our proposal does not rely on a merger to induce this age gap. We predict a $\sim300\,\Myr$ gap in stellar ages at a fixed \FeH{} where the $\alpha$-bimodality is prominent ($\FeH\lesssim-0.2$).}
\end{abstract}
  
\keywords{Milky Way disk (1050) --- Milky Way Galaxy physics (1056) --- Milky Way formation (1053), Hydrodynamical simulations (767) --- Post-starburst galaxies (2176) --- Galaxies (573) --- Starburst galaxies (1570)}

\section{Introduction} \label{sec:intro}
% What leads to distribution of heavy elements in a galaxy
Many elements heavier than hydrogen are produced through nuclear fusion in compact objects such as supernovae, dying low mass stars, and neutron star-neutron star mergers \citep[e.g.][]{2023A&ARv..31....1A}. By necessity, stars inherit the constituitive properties of the gas from which they formed. Moreover, the surface abundance of most elements for most stars do not change over most of their lifetime. \edit1{By analyzing the surface abundances of stars, we can reconstruct the historical gas-phase composition of a galaxy.}

The enrichment of the gas-phase of a galaxy is determined by a complicated combination of physical processes - stellar evolution and supernovae, gas accretion, mergers, gas outflows from stellar and active galactic nuclei (AGN) feedback, metal mixing and diffusion, etc. Because the processes which give rise to this distribution are complex, there is almost certainly some structure in the stellar abundance distribution for every galaxy. However, it has only been definitively measured in the Milky Way, \edit1{with conflicting claims of detection \citep{2023ApJ...956L..14K} and nondetection \citep{2024IAUS..377..115N} in M31.}

% Why alpha-elements are important
The distribution of elemental abundances is a high dimensional space \citep[e.g., 32 elements in][]{2024ApJ...961L..41J}. However, this space is highly degenerate, and so the effective number of dimensions is much smaller -- even possibly compressed to just \FeH{} and age \citep{2019ApJ...883..177N}. Two elements have received particular interest - Fe and elements produced by the $\alpha$-process. Type Ia and Type II supernovae are the main contributors of Fe and $\alpha$ enrichment. Fe is broadly produced in both types, and so its abundance is a proxy for the total metallicity of a star. On the other hand, $\alpha$-elements are mainly produced in Type II supernovae. The ratio of $\alpha$-elements to Fe (\alphaFe{}) is then a measure of the relative contributions of Type Ia and II SNe to the enrichment of a parcel of gas, \edit1{which typically declines with time \citep{1979ApJ...229.1046T,1986A&A...154..279M}}. It has therefore become common to compress the high-dimensional abundance space to the two dimensional \alphaFe{}-\FeH{} plane.

% Observed chemical bimodality in the Milky Way
\edit1{These abundances can be used to decompose the Milky Way's disk, which has a long history dating back to the work of \citet{1983MNRAS.202.1025G}, who noted that the vertical distribution of stellar altitudes is well-fit by a double exponential. This led naturally to a ``thin'' and ``thick'' disk, whose membership can be reasonably determined through kinematics \citep[e.g.][]{2003A&A...410..527B}. It was quickly realized that the thick disk is more $\alpha$-enhanced than the thin disk \citep{1996ASPC...92..307G,1998A&A...338..161F}.}

\edit1{Later studies showed that the disk could be decomposed into high- and low-$\alpha$ sequences without kinematic selection \citep{2011A&A...535L..11A,2012A&A...545A..32A}.\footnote{\citet{2003A&A...410..527B} briefly noted that the thin and thick disk seemed to not overlap in chemistry.} The high-$\alpha$ sequence is older, more centrally compact, and more vertically extended than the low-$\alpha$ sequence \citep{2013A&A...560A.109H,2024IAUS..377..115N}. Although the thick disk is more $\alpha$-enhanced than the thin disk, it is not immediately obvious that the chemical and kinematic separations arise from the same physical process \citep[or that they even exist, see][]{2012ApJ...751..131B}.}

% Different explanations of bimodality
Naturally, many different processes that could lead to structure in the abundance plane have been discussed in the literature. An early explanation of the bimodality is \edit1{based on} the two-phase gas infall model \citep{1997ApJ...477..765C,2009IAUS..254..191C,2017MNRAS.472.3637G,2019A&A...623A..60S}. In this model, the thick disk first forms rapidly from an initial infall of gas. Because the typical SFR is high, these stars are $\alpha$-enhanced. In some variants, star formation halts completely before a second supply of pristine gas falls into the Galaxy \citep[][and references therein]{2024arXiv240511025S}. This dilutes the gas supply from which the thin disk forms more gradually, \edit1{creating a loop feature in the abundance plane}. The thin disk is then more $\alpha$-poor because its associated SFR is lower, \edit1{and in certain scenarios two chemically distinct disks are formed.}

A later argument by \citet{2021MNRAS.501.5176K} asserts that the two sequences follow from two phases of gas infall, except driven by stellar feedback instead of cosmological inflow. An initial bursty phase follows from the direct collapse of the gaseous halo. The disk has a high SFR leading to the formation of the high-$\alpha$ sequence. Feedback then halts the inflow, and a slower accretion of high-angular momentum and metal-rich gas commences, forming the low-$\alpha$ sequence.

Another mechanism to generate structure in the abundance plane was pointed out by \citet{2009MNRAS.396..203S}, further developed by \citet{2021MNRAS.507.5882S,2023MNRAS.523.3791C}, and explored by \citet{2011ApJ...737....8L,2021MNRAS.508.4484J}. This model claims that, since stars are thought to migrate from their birth radius, there will be stars throughout the entire disk that formed in the inner disk. These $\alpha$-enhanced stars will then form the high-$\alpha$ sequence. This model and its variants also match some chemodynamic properties of the disk. One salient feature of these models is that the bimodality can result from a smooth star formation history.

Yet another explanation, which also invokes an internal process, is that the formation of clumps at high redshift are responsible for both the chemistry and dynamics of the high-$\alpha$ sequence \citep{2019MNRAS.484.3476C,2020MNRAS.492.4716B,2021MNRAS.502..260B,2023ApJ...953..128G}. Instabilities are thought to form clumps in gas-rich disks, and such clumps are seen at intermediate redshifts \citep[$z\sim2$;][]{2005ApJ...627..632E,2007ApJ...658..763E}. These clumps then self-enrich, forming $\alpha$-enhanced stars. The high-$\alpha$ sequence stops forming once the gas fraction is low enough for the instabilities to no longer arise. This model predicts that the high-$\alpha$ and low-$\alpha$ sequences form simultaneously.

Next, we turn to models which argue the bimodality results from some external influence. Early arguments were made that both the $\alpha$-enhancement of the disk and the thickening of the disk can result from gas-rich mergers \citep{2004ApJ...612..894B,2005ApJ...630..298B,2007ApJ...658...60B,2010MNRAS.402.1489R}.\footnote{See also \citet{2009MNRAS.400.1347C} for an argument invoking semi-analytic models.} These mergers lead to an enhanced SFR which leads to the $\alpha$-enhancement of the thick disk, \edit1{with \citet{2015A&A...578A..87S} being the first to attempt to explain abundance substructure with a merger.}

In cosmological simulations, which naturally include early gas-rich mergers, the situation is not as clear. Early work by \citet{2012MNRAS.426..690B} found a general separation between the thin and thick disk, though other authors found a smooth evolution \citep{2013A&A...558A...9M}. \citet{2018MNRAS.474.3629G} found what they referred to as a chemical dichotomy, and argued that it can come from either gas-rich mergers as described before or a ``compaction'' of the disk (we will return to this point in Section~\ref{ssec:cosmo}). Other authors highlight the metal content of the infalling gas, stating that the metal-poor gas associated with satellites can suddenly dilute or reset the disk's metallicity \citep{2020MNRAS.491.5435B,2024MNRAS.528L.122C}. This interpretation can also be understood in the framework of the two-infall models.

The merger explanation of the bimodality is highly synergistic with our picture of the hierarchical assembly of the stellar halo \citep{2005ApJ...635..931B}. Indeed, there is strong evidence that the Milky Way underwent a significant merger with the so-called Gaia-Sausage-Enceladus satellite \citep[GSE;][]{2018MNRAS.478..611B,2018Natur.563...85H,2020ApJ...901...48N}. This merger is thought to have occurred $\sim8-10\,\Gyr$ ago \citep[see also][]{2020ApJ...897L..18B}. \edit1{A merger origin of the abundance bimodality is also attractive because it can simultaneously explain the origin of the kinematic thin and thick disk \citep{1985AJ.....90.2015G,1986ApJ...309..472Q,1993ApJ...403...74Q}.}

Claims in the literature on the stellar mass of GSE vary widely. Early estimates argued from $6\times10^8$ up to even $10^{10}\,\Msun$ \citep{2018MNRAS.478..611B,2018Natur.563...85H,2019MNRAS.484.4471F,2019MNRAS.487L..47V,2019MNRAS.488.1235M,2020MNRAS.493.5195D,2020MNRAS.497..109F}. Later estimates have been more conservative ranging from a mass of $2.7\times10^8\,\Msun$ to $10^9\,\Msun$ \citep{2019MNRAS.482.3426M,2020MNRAS.492.3631M,2020MNRAS.498.2472K,2021ApJ...923...92N,2022AJ....164..249H}, and even as low as $1.5\times10^8\,\Msun$ \citep{2023MNRAS.526.1209L}.

\edit1{In this work, we propose that a brief $\sim300\,\Myr$ interruption in the formation of stars at a given metallicity can lead to the formation of an $\alpha$-abundance bimodality at that metallicity. It is not common to study the star formation rate at a specific metallicity, but dividing the stellar population into narrow abundance populations can be a powerful tool \citep[e.g.][]{2012ApJ...753..148B,2012ApJ...751..131B}.}

\edit1{Our proposal assumes that the gas phase's \alphaFe{} is declining sufficiently rapidly at the time of the interruption. Using a set of idealized simulations which mimic the $z\sim2$ merger between the Milky Way and GSE, we show that such a merger can drive the formation of this gap and thus the bimodality. While we demonstrate this mechanism in the context of a merger scenario, it is important to note that our proposal does not inherently require a merger to induce this metallicity-dependent star formation gap. This scenario predicts a $\sim300\,\Myr$ gap in stellar ages at metallicities where the bimodality exists ($\FeH\lesssim-0.2$).}

In Section~\ref{sec:methods}, we describe our setup. \edit1{In Section~\ref{sec:results}, we present in detail the main results of two example simulations before expanding our results to the full suite.} In Section~\ref{sec:discussion}, we discuss and interpret our results, as well as connections to previous and future work, before concluding in Section~\ref{sec:conclusion}. Throughout this work we refer to the standard native time unit $\kpc/\left(\kms\right)$ as \Gyr{} for convenience.

\section{Methods}\label{sec:methods}
\subsection{Isolated Setup}\label{ssec:iso_setup}
\begin{figure}
    \centering
    \includegraphics[width=\columnwidth]{mass_size.pdf}
    \caption{The mass and size evolution of the central (Milky Way, blue) and satellite (GSE, orange) galaxies simulated in isolation. The mass is taken to be the stellar mass within twice the half-mass radius, and the size is taken to be the half-mass radius. In the upper panel, we also show as a horizontal line the mass of the Milky Way's disk and GSE from the best-fit model of \citet{2021ApJ...923...92N}. This comparison is taken to be made at $3\,\Gyr$ (vertical dashed line), our proxy for $z\sim2$. A precise match is not attempted given the wide ranging uncertainties.}
    \label{fig:mass_size}
\end{figure}

We use a modified version of the \texttt{MakeNewDisk} variant described in \citet{2023MNRAS.tmp.2070B}. In isolation, each of the central and satellite galaxies are a compound halo setup, with a \citet{1990ApJ...356..359H} dark matter halo and a gaseous halo with a $\beta$-profile:
\begin{equation*}
\rho = \rho_0 \left[1 + \left(\frac{r}{r_c}\right)^2\right]^{-\frac{3\beta}{2}}
\end{equation*}
The total mass within the virial radius is kept fixed, and the mass of the dark matter halo and central density of the gaseous halo are chosen to satisfy a given baryon fraction $f_b$ \edit1{within the virial radius}. The dark matter halo is initialized to be in gravitational equilibrium with the total potential. The gaseous halo is in gravito-hydrostatic equilibrium, where the temperature is allowed to vary as a function of radius. The azimuthal velocity of the gaseous halo is given as a fraction of the circular velocity. There is no initial stellar disk or bulge, and the gas is initially metal-free. Thus, all star particles and metals are formed self-consistently.

\begin{figure}
    \centering
    \includegraphics[width=\columnwidth]{orbit.pdf}
    \caption{The orbits of the bimodal (blue) and unimodal (orange) simulations. The upper and middle panels show the orbit in the $x$-$y$ and $x$-$z$ planes, respectively. The bottom panel shows the separation distance as a function of time. The orbit begins retrograde but then radializes after the first pericentric passage. The satellite then coalesces quickly after the second pericentric passage, after $\sim2\,\Gyr$ of evolution. \edit1{A blue dashed line is shown for the bimodal simulation at $2.8\,\Gyr$, the time of a star formation gap at $\FeH=0$ (see Figure~\ref{fig:before_after}). An orange dashed line is shown for the unimodal simulation (which has no gap) at the same time after its second pericentric passage ($2.3\,\Gyr$).}}
    \label{fig:orbit}
\end{figure}

We used the fiducial halos in \citet{2021ApJ...923...92N} as a starting point for each galaxy. We then manually varied the different model parameters until we arrived at a setup that resulted in reasonable galaxies as determined by their stellar mass. For the central (Milky Way) galaxy, we set $M_{200}=5\times10^{11}\,\Msun$, $c_{200}=4.1$, $\beta=0.8$, $r_c=9\,\kpc$, $f_b=0.08$, and $v_{\phi}/v_{\textrm{c}}=0.2$, where $c_{200}$ is the concentration and $v_{\phi}/v_{\textrm{c}}$ is the azimuthal velocity of the gaseous halo as a fraction of the local circular velocity. For the satellite (GSE) galaxy, we set $M_{200}=2.2\times10^{11}\,\Msun$, $c_{200}=4.33$, $\beta=0.8$, $r_c=6.5\,\kpc$, $f_b=0.06$, and $v_{\phi}/v_{\textrm{c}}=0.4$.

We used a mass resolution of $6\times10^4\,\Msun$ for the gas and $3\times10^5\,\Msun$ for the dark matter. This is closest to a level~4 resolution in the AURIGA simulations \citep{2017MNRAS.467..179G}, and is about $0.7\times$ the mass resolution of TNG50-1 \citep{2019MNRAS.490.3234N,2019MNRAS.490.3196P}. All collisionless particles have a fixed softening length of $40\,\pc$. The gas has a softening length $2.5\times$ the cell size, with a minimum size of $10\,\pc$. Snapshots were saved at intervals of $25\,\Myr$.

The stellar mass build-up of our Milky Way-like and GSE-like galaxies is given in Figure~\ref{fig:mass_size}. The upper panel shows the stellar mass history. We attempt to match the expected mass of the present-day thick disk \citep[$\sim6\times10^9\,\Msun$, horizontal blue dashed line][]{2016ARA&A..54..529B} at an evolution time of $\sim3\,\Gyr$ (corresponding to $z\sim2$, vertical dashed gray line).\footnote{Of course, this neglects the significant mass contribution of the bulge, which presumably formed earlier. However, our setup does not form a strong spheroidal component. Using the trick in e.g. \citet{2022MNRAS.515.1524Z}, we take the bulge mass to be twice the counterrotating stellar mass. At $3\,\Gyr$ in the isolated Milky Way-like galaxy, the bulge mass is $\sim7\times10^{8}\,\Msun$, or $\sim13\%$ of the total mass. The Milky Way's bulge is  $\sim1.5\times10^{10}\,\Msun$, although there is strong debate about just how much of the bulge is a classical bulge which formed before the disk \citep{2016ARA&A..54..529B}. In any case, we did not attempt to match any particular property of the bulge, though one could promote bulge formation by reducing the rotation of the gas in the inner region.} We get reasonably close at $\sim5\times10^9\,\Msun$ (blue line). For GSE, we use the best-fit mass from the $N$-body simulations of \citet{2021ApJ...923...92N} -- $5\times10^8\,\Msun$ (horizontal dashed orange line). For this, we slightly overestimate at $\sim6\times10^8\,\Msun$ (orange line).

As for the galaxy sizes, there is significant spread amongst the real galaxy population, and the sizes are thought to be influenced by the merger history not present in our setup \citep[e.g.][]{2014ApJ...788...28V}. We note that the sizes of each simulated galaxy (lower panel) are within the range of observed galaxy sizes. For the Milky Way, we know the thick disk has scale length of $\sim2\,\kpc$, which converts to a half-mass radius of $\sim3.36\,\kpc$. At $\sim3\,\Gyr$, our Milky Way-like galaxy has a half-mass radius of $\sim2\,\kpc$. Curiously, after $3\,\Gyr$, the size of the Milky Way-like galaxy continues to grow while the GSE galaxy's size remains constant for the duration of the simulation.

\subsection{Orbital Configuration}\label{ssec:orbit_setup}
In order to combine the galaxies, we follow \citet{2021ApJ...923...92N}, and place the satellite on a retrograde orbit. In the fiducial simulation of \citet{2021ApJ...923...92N}, the satellite is placed at the virial radius ($R_0=129\,\kpc$), with the virial velocity ($V_0=129\,\kms$), and with a circularity of $\eta=0.5$. To test minor changes to the orbit, we ran a grid of simulations with $\pm10\%$ in each the starting radius and velocity, and $\pm0.1$ in the circularity, for a total of $27$ simulations. We performed each simulation for a duration of $8\,\Gyr$, and used \texttt{FOF} and \texttt{SUBFIND} in order to identify substructure \citep{2005Natur.435..629S,2009MNRAS.399..497D}.

Some of the simulations in this orbital grid resulted in bimodal abundance distributions, while some had little to no structure in the abundance distribution plane. \edit1{We will first study two representative simulations in detail chosen based on their structure in the abundance plane as shown in Figure~\ref{fig:fig1}, one which we refer to as bimodal and one as unimodal. For the bimodal simulation, we chose the simulation with $R_0=129\,\kpc$, $V_0=142\,\kms$, and $\eta=0.4$. For the unimodal simulation, the parameters are the same except that $V_0=116\,\kms$. These simulations will later be identified as having the highest and second lowest bimodality score $\mathcal{B}$. We will then examine the full simulation suite, and show the detailed abundance plane for the full suite5 in Appendix~\ref{app:allmerge}.}

We show the bimodal and unimodal simulations' orbits in Figure~\ref{fig:orbit}. \edit1{We use a shrinking spheres center of mass method to identify the centers of the central and satellite galaxy \citep[e.g.,][]{2003MNRAS.338...14P}.\footnote{The position of the minimum potential particle in each substructured identified by \texttt{SUBFIND} is used as the starting guess, and we use an initial/final radius and step factor of $10\,\kpc$, $5\,\kpc$, and $0.9$, respectively.}} The upper and middle panels show the orbits in the $x$-$y$ and $x$-$z$ planes, respectively. The lower panel shows the separation distance as a function of time. The orbit is initially retrograde, but quickly radializes after the first pericentric passage. Coalescence occurs rapidly after the second pericentric passage at $\sim2\,\Gyr$, and \texttt{SUBFIND} ceases to recognize the satellite as a separate subhalo.

\subsection{Feedback and Enrichment Model}\label{ssec:gfm}
Our feedback model is \edit1{a variant of the Illustris TNG model} \citep{2013MNRAS.436.3031V,2017MNRAS.465.3291W,2018MNRAS.473.4077P}. In this model, gravity and magnetohydrodynamics are solved using a \citet{1986Natur.324..446B} tree coupled to a second order finite volume fluid solver in AREPO \citep{2010MNRAS.401..791S,2016MNRAS.455.1134P}. Stellar feedback is included through a subgrid wind particle model \citep{2003MNRAS.339..289S}. AGN feedback follows a dual kinetic and thermal mode for low- and high-accretion rates \citep{2017MNRAS.465.3291W}, though in our setup the AGN is only ever in the high-accretion mode. The central galaxy is seeded with a black hole with the typical seed mass ($8\times10^5\,\Msun$). 

In this work, we made some simplifications to this model in order to aid interpretation. First, we ignore magnetic fields. This was motivated by an initial desire to understand the CGM accretion rates in terms of idealized cooling flow solutions, but we did not revisit turning them back on. In any case, it is not clear if the magnetic fields would be realistically generated given our initial setup. Second, we use a gentler wind feedback model as described in \citet{2019MNRAS.489.4233M}. Because our setup includes both an initially steep central potential and no steady-state disk, a stronger feedback model would require a higher central gas density to achieve a reasonable SFH which introduced its own set of pathological instabilities.

In this model, star particles synthesize elements through three different channels for which we cite the relevant yield tables: SNe Ia \citep{1997NuPhA.621..467N}, SNe II \citep{1998A&A...334..505P,2006ApJ...653.1145K}, and AGB stars \citep{2010MNRAS.403.1413K,2014MNRAS.437..195D,2014ApJ...797...44F}. Each star particle, \edit1{which is modeled as a simple stellar population}, continuously injects metals into its surroundings in the following sequence\footnote{The kinetic/thermal feedback component is handled through the wind generation, which is completely separate in this model.}:
\begin{enumerate}
    \item $t\lesssim10\,\Myr$: no metal injection as the first supernova ($M\sim100\,\Msun$) has not gone off
    \item $10\,\Myr \lesssim t \lesssim 40\,\Myr$: metal injection as $8\,\Msun<M<100\,\Msun$ stars die as Type II SNe
    \item $t\gtrsim40\,\Myr$: metal injection from Type Ia SNe and AGB stars
\end{enumerate}
There are a few things to note about this model: (1) The exact timings are metallicity-dependent. (2) The \MgFe{} of ejected gas from Type II SNe is mass/time-dependent, with more massive stars contributing more Mg than less massive stars. (3) In a Hubble time, type II SNe contribute the vast majority of Mg ($\sim10\times$ AGB and $\sim100\times$ Type Ia SNe). Type Ia and Type II SNe contribute approximately equal amounts of Fe (each $\sim3\times$ AGB). See Figure~1 from \citet{2018MNRAS.473.4077P}. (4) The number of Type Ia SNe is greater for a younger stellar population, with a power law relationship $\propto \left(t/\tau_8\right)^{-1.12}$, where $\tau_8=40\,\Myr$ is the lifetime of an $8\,\Msun$ star.

\subsection{Observed Abundances}\label{ssec:obs_abund}
Our aim in this work is to demonstrate the feasibility of a mechanism for structure formation in the abundance plane. We are only making a qualitative comparison to data. Therefore, we use the ASPCAP DR17 catalog of stellar abundances \citep[][J.A.~Holtzman et al., in preparation]{2016AJ....151..144G}, which is publicly available, well-established, and widely used.

We \edit1{applied quality cuts and restricted our sample to giants}, requiring:
\begin{itemize}[noitemsep]
    \item $\textrm{SNR} > 200$,
    \item $\textrm{VSCATTER} < 1\,\kms$,
    \item STARFLAG not set,
    \item $\varpi/\sigma_{\varpi} > 1$,
    \item $\log{g} < 3.5$,
    \item $\sigma_{\log{g}} < 0.2$,
\end{itemize}
where $\varpi$ is the parallax. We use the parallax, proper motion, and radial velocity from Gaia EDR3 \citep{2016AA...595A...1G,2021AA...649A...1G,2021AA...649A...2L,2021AA...653A.160S}.

\edit1{We next make a solar neighborhood selection of stars based on their angular momenta.} We assume the solar radius and azimuthal velocity are $R_0=8\,\kpc$ and $V_0=220\,\kms$ \citep{2016ARA&A..54..529B}, and select stars which have $L_z$ within $10\%$ of the solar angular momentum. We further require that $\left|z\right| < 3\,\kpc$. As is typically done, we use \FeH{} as an indicator of the total metallicity of a star. We use Mg alone as a representative of the $\alpha$-elements.

\edit1{For stellar ages, we used the APOKASC-3 catalog \citep{2025ApJS..276...69P}. This catalog uses a combination of APOGEE spectroscopic parameters and \textit{Kepler} time series photometry to compute astroseismic ages. Using only stars with $25\%$ age uncertainties (taken as the maximum of the upper and lower uncertainty), we cross-match this catalog to our larger sample from ASPCAP which results in a sample of 2525 stars.}

\subsection{Solar Neighborhood in Simulations}\label{ssec:solarneigh}
When comparing galaxy simulations to the observed solar neighborhood, some ambiguity arises in how to make a ``solar neighborhood-like'' selection of star particles. Naturally, this selection is dependent on the posed question, which in this work is the formation of the abundance bimodality. The Sun is known to sit near the end of the thick disk, where the thick and thin disk have comparable surface densities \citep[the ratio of thick-to-thin is $\sim12\%$][]{2016ARA&A..54..529B}. As a result, the abundance bimodality appears most strongly near the Sun -- further inwards the high-$\alpha$ sequence is more dominant and further outwards the high-$\alpha$ sequence vanishes \citep[e.g.,][]{2015ApJ...808..132H}.

We mimic our selection of the solar neighborhood by also making a cut in angular momentum. However, in the simulation, the high-$\alpha$ disk is more compact than in the Galaxy. Therefore, in order to strike a balance between the low-$\alpha$ and high-$\alpha$ disks, we used an angular momentum cut which is $20\%$ that of our assumed solar angular momentum. In particular, we select all star particles with angular momenta within $30\%$ of $0.2\times8\,\kpc\times220\,\kms$ -- as well as requiring $\left| z \right| < 3\,\kpc$. This corresponds to roughly selecting star particles with radii between $2$ and $5\,\kpc$.

\subsection{Bimodality Score}\label{ssec:bim_score}
\edit1{Given the modest size of our suite, some method for scoring the degree of bimodality for a given 1D distribution is desirable. Tests of whether a distribution is bimodal or unimodal exist -- e.g., the Hartigan dip test \citep{10.1214/aos/1176346577}, but they lack the ability to rank order based on ``bimodalitiness.'' In order to do this, we fit a given distribution of \alphaFe{} as a two-component gaussian mixture model. The bimodality score $\mathcal{B}$ is then computed as
\begin{equation}\label{eq:bimscore}
\mathcal{B} = \frac{\left|\mu_1-\mu_2\right|}{\sqrt{\sigma_1^2-\sigma_2^2}}\times w(A_2, t, k)\textrm{,}
\end{equation}
where $(\mu_1, \sigma_1)$ is the mean and standard deviation of the higher weighted mode, $(\mu_2, \sigma_2)$ likewise for the less weighted mode, $A_2$ is the amplitude of the less weighted mode, and $w$ is a penalty function with parameters $t$ and $k$ given by
\begin{equation}\label{eq:bimscore_penalty}
w(A, t, k) = \left[1+e^{-k\left(A-t\right)}\right]^{-1}\text{.}
\end{equation}
We take $t$ and $k$ to be $0.1$ and $20$, respectively.}

\edit1{This metric effectively measures the distance between two modes normalized by their variances, with a penalty when one mode is not highly weighted. These choices were made in order for the score to match by eye which distributions appear to have one as opposed to two peaks, and was found to empirically be more reliable than other statistics like peak-to-trough ratios or mode overlap. As demonstrated in Section~\ref{ssec:full_sim_suite}, a score of $\mathcal{B}>2.25$ appears to select the bimodal populations.}

\section{Results}\label{sec:results}
\subsection{Surface Density Projections}\label{ssec:projections}
Before dissecting the simulated galaxies in detail, we first take a look at the surface density projections of the gas and stars in the central region, situated on the central galaxy in the bimodal simulation. This is shown in Figure~\ref{fig:projections}. In each grouping of four, the bottom and top rows show the face-on and edge-on views, respectively. The left and right columns show the gas (blue) and stellar (orange) surface density, respectively. Each panel is oriented with respect to the stellar angular momentum of the final snapshot ($t=8\,\Gyr$). The angular momentum is computed using all star particles within the half-mass radius.

We can see that the galaxy grows in size after the merger. Additionally, the galaxy's orientation continues to change even within the last $3\,\Gyr$. The final galaxy is oriented $\sim126\degree$ with respect to the initial orientation of the central galaxy's gas halo. The dynamical and kinematic consequences of such a gas-rich merger is beyond the scope of our current work.

\begin{figure*}
  \centering

    \gridline{\fig{all_L30_frame60.pdf}{0.3\textwidth}{}
              \fig{all_L30_frame70.pdf}{0.3\textwidth}{}
              \fig{all_L30_frame80.pdf}{0.3\textwidth}{}
              }
    \gridline{\fig{all_L30_frame90.pdf}{0.3\textwidth}{}
              \fig{all_L30_frame100.pdf}{0.3\textwidth}{}
              \fig{all_L30_frame110.pdf}{0.3\textwidth}{}
              }
    \gridline{\fig{all_L30_frame120.pdf}{0.3\textwidth}{}
              \fig{all_L30_frame200.pdf}{0.3\textwidth}{}
              \fig{all_L30_frame320.pdf}{0.3\textwidth}{}
              }
  % \includegraphics[width=\textwidth]{surfdens.pdf}
  \caption{Frames from a movie showing a surface density projection of the bimodal simulation over time. In each frame, the left/right (blue/orange) column shows the gas/star surface density. The upper/lower panels show the edge-on and face-on view. Every panel is oriented with respect to the final ($t=8\,\Gyr$) snapshot. The side-length of each panel is $30\,\kpc$, and the image is a projection through a box with the same side-length. The colormap for the gas ranges from $1$ to $10^2\,\Msun/\pc^2$, while for the stars ranges from $1$ to $10^4\,\Msun/\pc^2$. (A full movie is \href{https://drive.google.com/file/d/1RKpFpnFIpl7XYAX5lvF7CyaPVO5dS3EO/view?usp=drive_link}{here}.)}
  \label{fig:projections}
\end{figure*}

\subsection{Abundance Distribution}\label{ssec:abundplane}
In Figure~\ref{fig:fig1}, briefly discussed in Section~\ref{sec:intro}, we show the abundance distribution of the Milky Way as well as two of our idealized merger simulations in the upper panels. A number of our idealized simulations exhibit either a bimodal or unimodal abundance distribution, and so we have selected two representative examples. The simulation labeled bimodal had orbital parameters of $R_0=142\,\kpc$, $V_0=116\,\kms$, and $\eta=0.4$. The unimodal simulation is identical, except that $R_0=129\,\kpc$. The bimodal and unimodal labels are of the outcome of the simulation, and do not reflect any particular choice made in their setup.

There are, of course, differences between the bimodal simulation and the Milky Way. First, the scaling in \MgFe{} is different -- in the simulation, the low-$\alpha$ sequence lies at $\sim0.2$, while in the Milky Way it is at about $\MgFe\sim0$. Second, in the Milky Way the high-$\alpha$ sequence neatly joins the low-$\alpha$ sequence at $\FeH\sim0$, while in the simulation the two actually diverge more at higher \FeH{}. Finally, at high \FeH{} the distribution bends upwards while in the simulation it bends downwards.

In the lower panels of Figure~\ref{fig:fig1}, we show the distributions of \MgFe{} at different fixed \FeH{}. The (blue, orange, red, green) lines show the \MgFe{} distribution at a \FeH{} of ($-0.5$, $-0.25$, $0$, $0.25$), in bins of width $0.05\,\dex$. The distributions of \MgFe{} are offset (but not rescaled) so that they do not overlap. Here, the bimodality seen in the Milky Way is quite striking at lower metallicities. The peaks are well-separated, by $\sim0.2\,\dex$. In the bimodal simulation, the distribution is still clearly bimodal, but the peaks are less well-separated, by $\sim0.1\,\dex$. In the unimodal simulation, there is a hint of some structure at $\FeH \gtrsim 0.25$, but there is not a strong multimodal structure.
\begin{figure*}
  \centering
  \includegraphics[width=\textwidth]{figure1.pdf}
  \caption{The abundance bimodality seen in the Milky Way can be reproduced in some idealized merger simulations. In the upper panels, we show the distribution of stars in the \MgFe{}-\FeH{} plane. The lower panels show the distribution of \MgFe{} at a fixed \FeH{} bin of width $0.05\,\dex$. The colors indicate the fixed \FeH values, which are $-0.5$, $-0.25$, $0$, and $0.25$. The left column shows the observed distribution in the Milky Way from ASPCAP DR17 \citep[][J.A.~Holtzman et al., in preparation]{2016AJ....151..144G}, while the right two columns show two idealized merger simulations. The idealized merger simulations are nearly identical, except that in the bimodal simulation the satellite has a starting \edit1{velocity of $142\,\kms$}, while in the unimodal simulation it has a starting \edit1{velocity of $116\,\kms$}. The labels ``unimodal'' and ``bimodal'' are of the \textit{outcome} of the simulation, and do not reflect a particular choice in the setup. The Milky Way (left column) exhibits a strong bimodal distribution of \MgFe{} at various \FeH{}. The idealized merger simulation marked as bimodal (center column) also exhibits a bimodal distribution of \MgFe{}, though the structure is not as strongly defined. The idealized merger simulation marked as unimodal (right column) exhibits only weak structure, if any at all.}
  \label{fig:fig1}
\end{figure*}

\subsection{Build up of the Abundance Plane}\label{ssec:abundplane_build}
Next, we examine the build up of the abundance plane. First, in Figure~\ref{fig:before_after}, we show the abundance plane at various points along the merger. We have divided the history of the merger into ``before'' ($t < 1.5\,\Gyr$), ``during'' ($1.5\,\Gyr < t < 2.5\,\Gyr$), and ``after'' ($t > 2.5\,\Gyr$). These times have been chosen somewhat by the orbital history and somewhat by visually inspecting the simulation, and so these deliniations are, to an extent, arbitrary. Nonetheless, it is clear that the high-$\alpha$ sequence forms before and during the merger (middle panels), and that the low-$\alpha$ sequence forms after the merger (right panel). We show a dashed line given by $-0.1\times\FeH + 0.31$, which is a crude way of separating the high- and low-$\alpha$ sequence.

It is informative to examine the star formation history (SFH) split into the high- and low-$\alpha$ sequences at a fixed \FeH{}, shown in Figure~\ref{fig:before_after_sfh}. The high- and low-$\alpha$ sequences (blue and orange) are defined as the star particles lying above and below the dashed line in Figure~\ref{fig:before_after}, respectively. We show this only in a narrow bin of width $0.1\,\dex$ at $\FeH\sim0$. We chose this value of \FeH{} because here the valley of the bimodality is deepest (see Figure~\ref{fig:fig1}). One can see a very clear separation in time between the high- and low-$\alpha$ sequences, with the high-$\alpha$ sequence forming before the low-$\alpha$ sequence.\footnote{There is a small amount of low-$\alpha$ formation at the very end of the high-$\alpha$ sequence, but this can be understood as simply an inadequacy in our simple way of separating the two sequences. Alternatively, one could assert that the separation in time is the real delineation between the two sequences, and that the boundary in the abundance plane has only mild contamination.} Furthermore, there is a brief period of $\sim300\,\Myr$ between the two sequences where no stars have formed.

It can also be shown that the exact timing of this quiescent period is metallicity-dependent. We show the same SFH in Figure~\ref{fig:before_after_sfh_by_iron}, but at varying bins in \FeH{} of $-0.5$, $-0.25$, and $0$ (blue, orange, and red, respectively). For clarity, we do not separate the SFHs by the high- and low-$\alpha$ sequences, but have verified that the same demarcation arises as in Figure~\ref{fig:before_after_sfh}. Here, we can see that at $\FeH=-0.5$ the quiescent period begins $\sim250\,\Myr$ earlier than at $\FeH=0$.

\subsection{\alphaFe{} Behavior Associated with Quiescence}
We now examine more closely the natural suspicion that the quiescent phase in Figure~\ref{fig:before_after_sfh} is responsible for the bimodal structure in Figure~\ref{fig:before_after}. In Figure~\ref{fig:SFR_alpha}, we show the SFR (blue lines) of each of the bimodal and unimodal galaxies (left and right panels, respectively) as well as the median \MgFe{} of the gas (orange lines) in a bin centered on $\FeH{}=-0.1$ with width $0.05\,\dex$. The SFR is shown for the entire galaxy ($r<15\,\kpc$) while the median \MgFe{} is shown for $2\,\kpc<r<5\,\kpc$.\footnote{The solar neighborhood-like selection of stars (discussed in Section~\ref{ssec:solarneigh}) mostly selects stars at $2\,\kpc<r<5\,\kpc$. We choose to plot the median \MgFe{} of the gas at these radii, since it is roughly that gas which is forming stars that ends up in our sample. On the other hand, we choose to show the SFR for the entire galaxy, as the enrichment from that star formation affects the entire galaxy at this epoch.}

The merger appears to induce a starburst, where the SFR rises from $\sim3.5\,\Msunyr$ at $\sim1\Gyr$ to a peak of $\sim12\,\Msunyr$ at $\sim2\,\Gyr$. This starburst is then followed by a quiescent period, with a minimum SFR of $\sim1\,\Msunyr$ and a duration of $\sim500\,\Myr$. The dip in SFR is not as pronounced as it was when we plotted the SFH at a particular iron (Figure~\ref{fig:before_after_sfh}). This quiescent period is associated with a drop in \MgFe{} from $\sim0.35$ to $\sim0.2\,\dex$ in only $\sim250\,\Myr$. Once the SFR recovers $\sim500\,\Myr$ later, \MgFe{} again rises, but only to $\sim0.28\,\dex$.

This behavior is absent in the unimodal simulation (right panel) -- while there is a starburst, it does not display a period of suppressed star formation, and there is no associated drop in \MgFe{}. It does still have a period of enhanced star formation, but this is actually coincident with an enhancement in \MgFe{}. This sequence of events is further emphasized in Figure~\ref{fig:alpha_vs_tform}, which shows a transparent scatter plot of \MgFe{} as a function of formation time. In the bimodal simulation, there is a clear gap in \MgFe{} during the period of quiescence followed by the brief formation of $\alpha$-deficient stars. In the unimodal simulation, on the other hand, there is no gap and instead a brief formation of $\alpha$-enhanced stars.

\subsection{Cause of Quiescence}\label{ssec:cause_qui}
Why there is a quiescent period in the bimodal simulation but not the unimodal simulation probably has to do with some transient, stochastic property of the gas in and around the galaxy. In some states, the AGN is better able to quench the galaxy than in others. Verifying this further is beyond the scope of this work, but we show some evidence of the association in Appendix~\ref{app:cause_qui}.

This outcome suggests that a merger alone is not sufficient in order to generate a bimodality. Rather, the merger must induce a quenching phase. Whether this occurs depends both on the properties of the merger but also on the state of the central galaxy itself and whether it can launch AGN-driven winds that result in significant gas removal. Our work would argue for some stochasticity in whether a particular merger leads to such quenching.

\begin{figure*}
  \centering
  \includegraphics[width=\textwidth]{before_after.pdf}
  \caption{\textbf{The high-$\alpha$ sequence forms before the merger, the low-$\alpha$ sequence forms after the merger.} This plot shows the sequence of events leading to the build-up of the low- and high-$\alpha$ sequences for our fiducial bimodal simulation. We have separated the high- and low-$\alpha$ sequences by a dashed line at $-0.1\FeH + 0.31$, which was chosen by eye to lie in the trough. The left panel shows all star particles in our solar neighborhood cut. The middle left panel shows the star particles that form before the merger ($\tform < 1.5\,\Gyr$), which form a weak sequence of star particles at the lowest \FeH and highest \MgFe. The middle right panel shows the star particles that form during the starburst ($1.5\,\Gyr < \tform < 2.5\,\Gyr$). These star particles form the portion of the high-$\alpha$ sequence closest to the trough, and the density of star particless is higher than those that form before. The right panel shows the star particles which form after the merger ($\tform > 2.5\,\Gyr$). These star particles form almost entirely below the trough.}
  \label{fig:before_after}
\end{figure*}

\begin{figure}
  \centering
  \includegraphics[width=\columnwidth]{before_after_sfh.pdf}
  \caption{\textbf{At fixed metallicity, the low- and high-$\alpha$ sequence are cleanly separated in time by an intervening quiescent period.} Here we show the star formation history of the high- and low-$\alpha$ sequences. The blue (low-$\alpha$) and orange (high-$\alpha$) histograms correspond to the dashed line cut made in the \MgFe{}-\FeH{} plane in Figure~\ref{fig:before_after}, at a fixed metallicity of $\FeH=0$ with width $0.05\,\dex$. One can see that there is a nearly perfect separation in time between the formation of the high-$\alpha$ and low-$\alpha$ sequences by a quiescent period lasting $\sim300\,\Myr$. There is a small amount of contamination at the end of the high-$\alpha$ sequence which can be attributed to the inadequacy of the linear cut model of the two sequences.}
  \label{fig:before_after_sfh}
\end{figure}

\begin{figure}
  \centering
  \includegraphics[width=\columnwidth]{before_after_sfh_by_iron.pdf}
  \caption{\textbf{The timing of the quiescent period which divides the high- and low-$\alpha$ sequences is metallicity dependent.} The star formation history of stars at various fixed metallicities ($\FeH=0$, $-0.25$, and $-0.5\,\textrm{dex}$) in bins of width $0.05\,\dex$. As in Figure~\ref{fig:before_after_sfh}, there is a quiescent period period which separates the formation of the low- and high-$\alpha$ sequence. However, the timing of the quiescent period is metallicity-dependent. It occurs earlier at lower metallicities, by $\sim250\,\Myr$ for the metallicities shown in this plot.}
  \label{fig:before_after_sfh_by_iron}
\end{figure}

\begin{figure*}
  \centering
  \includegraphics[width=\textwidth]{SFR_alpha.pdf}
  \caption{\textbf{A global suppression of star formation is associcated with a decrease in \MgFe{}, which is seen in the bimodal simulation but not in the unimodal simulation.} Here, we show both the SFR of the central galaxy ($r<15\,\kpc$) and the median \MgFe{} for gas at $2\,\kpc<r<5\,\kpc$ at a fixed \FeH{} bin centered on $0$ with width $0.05\,\dex$. The left panel shows the bimodal simulation while the right panel shows the unimodal simulation. A vertical dashed line at $2\,\Gyr$ in each plot indicates the approximate time of the merger, but before complete coalescence. In the left panel, a shaded region from $2.5\,\Gyr$ to $2.75\,\Gyr$ indicates the approximate period of quiescence shown in Figure~\ref{fig:before_after_sfh}. This suppression of star formation is associated with a sudden drop in the median \MgFe{} of the gas. Neither the suppression of star formation nor the drop in \MgFe{} are seen in the unimodal simulation.}
  \label{fig:SFR_alpha}
\end{figure*}

\begin{figure}
  \centering
  \includegraphics[width=\columnwidth]{alpha_vs_tform.pdf}
  \caption{\textbf{In the bimodal simulation, a gap in stellar ages is briefly followed by the formation of $\alpha$-deficient star particles.} The formation times and \MgFe{} values of star particles in the bimodal (blue) and unimodal (orange) simulations. In the unimodal simulation, no gap is present and there is briefly the formation of $\alpha$-enhanced star particles at the same time. The \MgFe{} values of the unimodal simulation have been offset by $-0.3$ for clarity.}
  \label{fig:alpha_vs_tform}
\end{figure}

\section{Discussion}\label{sec:discussion}
We have demonstrated that a bimodal structure in the elemental abundance plane is associated with a brief quiescent phase in our idealized merger simulations. We now construct a possible framework for understanding the formation of the bimodality, discuss its relation to prior work, and then discuss predictions and connections to observations.

\subsection{Quiescence Leads to Bimodality}\label{ssec:formqui}
We executed a series of idealized merger simulations in which we modified the starting radius and velocity by $\pm10\%$ and the circularity by $\pm0.1$, for a total of $27$ simulations. The central and satellite galaxies, which are meant to resemble the Milky Way and GSE at $z\sim2$, are otherwise identical across the simulations. Some of these simulations induce a bimodality, while others do not. We have examined a representative of each scenario in detail.

We described a series of results in Section~\ref{sec:results} which give a natural explanation for the separation between the high- and low-$\alpha$ sequences in our simulations. First, a merger induces a starburst in the simulation. If this starburst does not induce a quiescent period, then the median \MgFe{} of the gas briefly increases before continuing along the typical evolutionary path (right panel of Figure~\ref{fig:SFR_alpha}). However, if there is a quiescent period, the median \MgFe{} drops for the duration of the quiescent period before recovering to the typical path (left panel of Figure~\ref{fig:SFR_alpha}).

This drop can be understood in terms of the mechanics of the galaxy formation model as described in Section~\ref{ssec:gfm}. When a star particle forms, it only takes $\sim40\,\Myr$ before the last Type II SN has gone off and its material enrichment is dominated by Fe from Type Ia SNe. As a result, during the quiescent phase which lasts a few hundred Myr, there is relatively little $\alpha$-element production. On the other hand, star particles are still able to generate Fe through Type Ia SNe which have a longer period of influence. The fact that the quiescent period is preceded by a starburst is important. Even though star particles produce Fe through the Type Ia channel for Gyr, younger star particles still produce much more Fe than older star particles. In the TNG model, the number of Type Ia SNe decays with time as $(t/\tau_8)^{-s}$, where $\tau_8=40\,\Myr$ is the lifetime of an $8\,\Msun$ star and $s=1.12$. Therefore, the burst of star formation at $t\sim2.25\,\Gyr$ is able to generously produce Fe starting at $2.5\,\Gyr$, when is indeed when the drop in \MgFe{} is the most steep.

\subsection{Connection to Cosmological Simulations}\label{ssec:cosmo}
As discussed in Section~\ref{sec:intro}, several authors have examined the formation of abundance plane structure in cosmological simulations. Of most interest to us is the zoom Au~23 in \citet{2018MNRAS.474.3629G}. This galaxy, one of six considered in their work, exhibits a bimodality that extends beyond the inner disk. The interpretation given by the authors is of a ``shrinking'' gaseous disk. This is equivalent to saying that the outer disk becomes depleted of gas. This shrinking of the disk, which occurs at $t_{\textrm{lookback}}\sim6\,\Gyr$, is associated with a dip in the SFR at that radius and a decrement in the median \alphaFe{} of $\sim0.05\,\dex$ (their Figure~2), which shortly after recovers. This sequence of events is more extended than in our work, but it strongly resembles the scenario in Figure~\ref{fig:before_after_sfh}.

\citet{2018MNRAS.477.5072M} found that Milky Way-like bimodalities are rare in EAGLE, occuring in $\sim5\%$ of galaxies. \citet{2021MNRAS.501..236D,2022MNRAS.515.1430D} showed that merger-induced quenching in zooms can occur in the EAGLE model \citep[see also][]{2017MNRAS.465..547P}. However, the situation may be different in the lower resolution large box. Furthermore, if the proposed starburst-quenching phase is driven by AGN feedback, then the outcome of any particular cosmological simulation with regards to the bimodality is intimately tied to its AGN model. Unfortunately, such models are highly uncertain \citep[e.g.][]{2022MNRAS.511.3751H}.

\citet{2023arXiv231016085K} explored the impact of quenching in the Magneticum Pathfinder suite. They found that galaxies which quench undergo a starburst followed by an AGN-driven quenching phase. In the post-starburst regime, they claim galaxies are $\alpha$-enhanced. We do find that the bulk stellar \MgFe{} is enhanced after the merger in our bimodal simulation compared to the isolated simulation, but only at the $\sim0.01\,\dex$ level.

\subsection{Infall Interpretations}\label{ssec:dilute}
In some previous work, it was reported that the bimodality is a consequence of a sudden deposition of metal-poor, $\alpha$-poor gas by a satellite or cosmological filament -- i.e., a ``dilution'' \citep{2020MNRAS.491.5435B,2021MNRAS.503.5846R}. The separation of the sequences follows from the rapidity of the dilution. This was elaborated upon by \citet{2021MNRAS.503.5868R} who described a zoom where the low-$\alpha$ disk forms out of a relatively pristine cosmological filament. This disk is inclined relative to the high-$\alpha$ disk, with the two disks later tidally realigning. The longer-standing two-infall class of models argue that the two sequences diverge due to two episodic accretion episodes, with some possible enrichment of the second episode arising from an associated satellite \citep{1997ApJ...477..765C,2009IAUS..254..191C,2017MNRAS.472.3637G,2019A&A...623A..60S}.

We have shown that minor changes to the orbit of our idealized merger can result in outcomes that are either bimodal or unimodal. The content of gas that is delivered to the system is nearly identical regardless of the orbit, and so the dilution interpretation is not applicable to our simulations. That being said, a removal of gas from the system either through star formation or through ejection could make dilution from infalling gas more efficient, so the two scenarios are not mutually exclusive.

It was recently elaborated by \citet{2024arXiv240511025S} that these models also argue for a star formation gap between the two accretion episodes \citep[see also][]{1996ASPC...92..307G,1998A&A...338..161F,2000A&A...358..671G,2015A&A...578A..87S,2020A&A...640A..81N}. This gap is starvation-driven and can last several \Gyr{}. The present work argues for a starburst-driven quiescence followed by a rapid rejuvenation, with the entire process taking less than $1\,\Gyr$ and the gap only lasting a few hundred \Myr{}. The physical origin and some details are different, but one can appreciate that the bimodality arises from the same physical process.\footnote{Compare Section~\ref{ssec:formqui} to the first key result in the Conclusions of \citet{2024arXiv240511025S}.}

\subsection{Observational Connections: High Redshift Quenching}\label{ssec:obshiz}
If the Milky Way underwent a starburst to quiescent to rejuvenation sequence, as we argue in this paper, then an an analogue at $z\sim2$ would be identified as a post-starburst galaxy (PSB). With abundance matching, we expect the Milky Way's total stellar mass to be $\sim10^{10.3}\,\Msun$ at $z\sim2$ \citep{2013ApJ...771L..35V}. A number of authors have explored PSBs and quiescent galaxies at slightly higher masses at $z\sim2$, with large advances in the post-JWST era.

First, PSBs are not uncommon. \citet{2023ApJ...953..119P} found that in massive galaxies ($M_* > 10^{10.6}\,\Msun$) the fraction of PSBs (inferred ages $< 800\,\Myr$) increases from $\sim2.7\%$ ($99/3655$) at $1.0 < z < 1.44$ to $\sim8\%$ ($89/1118$) at $2.16 < z < 2.5$ \citep[see also][]{2012ApJ...745..179W,2019ApJ...874...17B}. Later, \citet{2024arXiv240417945P} found that $\sim10\%$ of galaxies at $\sim10^{10.3}\,\Msun$ are quenched \citep[consistent with][]{2013ApJ...777...18M}, and $\sim30\%$ of their quiescent sample is a PSB at $z\sim2$. If these galaxies can be quickly rejuvenated, as the system studied in this work would suggest, then the total fraction of galaxies that go through a starburst-quenching phase may be higher.

Furthermore, \citet{2023arXiv231215012C} found that lower mass quiescent galaxies (towards $10^{10.3}\,\Msun$) tend to be younger and more disky, pointing to a merger driven scenario. \citet{2023arXiv231212207A} found that at even lower masses ($<10^{9.5}\,\Msun$) There is also evidence that the AGN quenching mechanism, which we believe to be responsible for the quenching in our system (Appendix~\ref{app:cause_qui}), is operating at these redshifts \citep[e.g.][and references therein]{2023arXiv230806317D,2024arXiv240417945P,2024arXiv240518685M,2024Natur.630...54B}.

Overall, the starburst-driven quenching mechanism proposed in this work is allowed by high-redshift observations.

\subsection{Direct Search for Quiescence}\label{ssec:obsqui}
Figure~\ref{fig:before_after_sfh_by_iron} indicates a very direct observational test of the mechanism proposed in this work: for disk stars at a given \FeH{}, there should be a gap of $\sim300\,\Myr$ in ages at $\sim8\,\Gyr$, though in the Milky Way the gap could be larger. With a survey of properly chosen stars, this gap could be directly measured with a modestly sized (few hundred) sample of old stars with age uncertainties of $\sim1\%$. To our knowledge, the best method at these ages is differential analysis of solar twins, which can provide an age uncertainty of $\sim5\%$ \citep[e.g.][]{2014ApJ...795...23B,2018MNRAS.474.2580S}. However, this has only been applied to stars with solar metallicity, where there is not a clear separation between the high- and low-$\alpha$ sequences (though a gap in ages may still be present). A sort of differential approach could be applied also at lower metallicities, which would just lack the absolute age calibration that the Sun provides.

The gap could also be indirectly probed by a larger sample of slightly less precise ages. Astroseismology appears to be the most promising avenue. Currently the largest sample is APOKASC2 \citep{2018ApJS..239...32P}, which has $\sim3400$ stars with ages measured to $<20\%$ precision.\footnote{Errors here taken to be the maximum of the upper and lower age estimates in the public APOKASC2 catalog.} The upcoming PLATO mission is looking to measure $\sim20\textrm{k}$ stars with ages measured to $<10\%$ precision \citep{2024arXiv240605447R}. Further work is needed to determine how well these surveys could constrain the gap.

In addition to the presence of the gap, our work also predicts that: (1) the timing of the gap is \FeH{} dependent, and that (2) stars which form immediately after the gap are $\alpha$-poor compared to stars which form slightly later (see Figure~\ref{fig:alpha_vs_tform}). However, both of these tests rely on the precise age estimates.

\subsection{Future Work}
A natural next step would be to extend the idealized simulations in this work to a wider range of orbits, galaxy properties, and feedback models. However, the unrealism of our setup, which aids interpretation, limits its applicability to the real universe. And so something along the lines of the genetic modification technique to explore various mergers as done in this work may be useful \citep{2016MNRAS.455..974R,2017MNRAS.465..547P}.

There is, of course, still great uncertainties in the stellar evolution models commonly adopted by different groups. Initial work on systematically exploring the stellar evolution parameters has been done by \citet{2017A&A...605A..59R,2021MNRAS.508.3365B}. Exploring these variations in the simulations presented in this work would be interesting, though exploring their interactions with brief quiescent periods in simpler chemical evolution models may be a better first step.

There is also the perennial problem of diffusion within the hydrodynamics solver. In purely Lagrangian solvers, there is no diffusion between resolution elements, while in Eulerian codes the diffusion can be quite high.\footnote{No galaxy formation simulation is fully Lagrangian since there must be, at a minimum, mass exchange between star particles and gas. Here we just mean that there is no mass exchange between gas elements.} AREPO limits the numerical diffusion by allowing the mesh to move in a quasi-Lagrangian manner, and using a second-order solver \citep{2010MNRAS.401..791S}. In FIRE-2 \citep{2018MNRAS.480..800H}, which uses the Lagrangian code GIZMO \citep{2015MNRAS.450...53H}, a subgrid turbulent metal diffusion model was used. It would be interesting to see how models with different diffusivity properties would relax or strengthen the necessity of a quiescent period to produce a bimodality.

\section{Conclusion}\label{sec:conclusion}
The \alphaFe{}-\FeH{} plane of stellar abundances is a record of the gas-phase abundances of the Galaxy. In this plane, a bimodality has now been definitively measured. Proposals for its formation include radial migration, particular gas infall scenarios, and galaxy mergers.

In this work, using idealized merger simulations, we have shown that (1) a starburst, followed by (2) a brief ($\sim300\,\Myr$) quiescent period, followed by (3) rejuvenation can lead to a bimodal abundance structure. We found that the starburst-to-quiescent period allowed for prodigious Fe-enrichment without any contemperaneous $\alpha$-enrichment. This sequence of events, somewhat resembling the life cycle of a phoenix, would predict that for stars at a given \FeH{}, there would be a $\sim300\,\Myr$ gap in stellar ages at $\sim8\,\Gyr$. Furthermore, stars which form directly after this gap would have lower \alphaFe{} than stars which form slightly later ($\sim1\,\Gyr$).

The bimodality in the \alphaFe{}-\FeH{} plane of stellar abundances points to deeper, unobserved structure. That structure is a record of the complicated processes that gave rise to the Galaxy. In the absence of highly precise stellar ages, that history must be largely inferred. There are many ways in which a galaxy could end up with the coarse-grained structure we can observe in the present day. We have argued in this work that a brief, starburst-driven quiescent phase is one of them.

\begin{acknowledgements}
We would like to thank Megan Bedell, Christopher Carr, Vedant Chandra, Charlie Conroy, Drummond B. Fielding, Lars Hernquist, Federico Marinacci, Rohan P. Naidu, Melissa K. Ness, Minjung Park, Vadim A. Semenov, and Turner Woody for helpful discussions. We would like to thank Filippo Barbani for kindly providing his version of \texttt{MakeNewDisk}. A.B. would like to thank Todd Phillips for helpful discussions.

This work has made use of data from the European Space Agency (ESA) mission {\it Gaia} (\url{https://www.cosmos.esa.int/gaia}), processed by the {\it Gaia} Data Processing and Analysis Consortium (DPAC, \url{https://www.cosmos.esa.int/web/gaia/dpac/consortium}). Funding for the DPAC has been provided by national institutions, in particular the institutions participating in the {\it Gaia} Multilateral Agreement.

Funding for the Sloan Digital Sky 
Survey IV has been provided by the 
Alfred P. Sloan Foundation, the U.S. 
Department of Energy Office of 
Science, and the Participating 
Institutions. 

SDSS-IV acknowledges support and 
resources from the Center for High 
Performance Computing  at the 
University of Utah. The SDSS 
website is www.sdss4.org.

SDSS-IV is managed by the 
Astrophysical Research Consortium 
for the Participating Institutions 
of the SDSS Collaboration including 
the Brazilian Participation Group, 
the Carnegie Institution for Science, 
Carnegie Mellon University, Center for 
Astrophysics | Harvard \& 
Smithsonian, the Chilean Participation 
Group, the French Participation Group, 
Instituto de Astrof\'isica de 
Canarias, The Johns Hopkins 
University, Kavli Institute for the 
Physics and Mathematics of the 
Universe (IPMU) / University of 
Tokyo, the Korean Participation Group, 
Lawrence Berkeley National Laboratory, 
Leibniz Institut f\"ur Astrophysik 
Potsdam (AIP),  Max-Planck-Institut 
f\"ur Astronomie (MPIA Heidelberg), 
Max-Planck-Institut f\"ur 
Astrophysik (MPA Garching), 
Max-Planck-Institut f\"ur 
Extraterrestrische Physik (MPE), 
National Astronomical Observatories of 
China, New Mexico State University, 
New York University, University of 
Notre Dame, Observat\'ario 
Nacional / MCTI, The Ohio State 
University, Pennsylvania State 
University, Shanghai 
Astronomical Observatory, United 
Kingdom Participation Group, 
Universidad Nacional Aut\'onoma 
de M\'exico, University of Arizona, 
University of Colorado Boulder, 
University of Oxford, University of 
Portsmouth, University of Utah, 
University of Virginia, University 
of Washington, University of 
Wisconsin, Vanderbilt University, 
and Yale University.

\end{acknowledgements}

\bibliography{ref}{}
\bibliographystyle{aasjournal}

\appendix
\section{Abundance Plane of All Simulations}\label{app:allmerge}
We show summary plots of the abundance planes of all simulations in our orbital grid in Figures~\ref{fig:allmerge0} to \ref{fig:allmerge8}.

\begin{figure*}
  \centering
  \includegraphics[width=\textwidth]{allmerge0.pdf}
  \caption{A summary of the abundance plane and star formation history of all simulations within the orbital grid. Each figure shows the outcome of a simulation at a fixed $R_0$ and $V_0$, varying $\eta$. The title of each column shows the $R_0$, $V_0$, and $\eta$ of that simulation, in order. The upper and middle rows replicate Figure~\ref{fig:fig1}, which show the distribution of stars in the abundance plane of \MgFe{}-\FeH{} as well as 1D histograms at a fixed \FeH{} of $-0.5$, $-0.25$, $0$, and $0.25$. The lower rows replicate Figure~\ref{fig:before_after_sfh_by_iron}, showing the star formation history at each \FeH{}.}
  \label{fig:allmerge0}
\end{figure*}

\begin{figure*}
  \centering
  \includegraphics[width=\textwidth]{allmerge1.pdf}
  \caption{A continuation of Figure~\ref{fig:allmerge0}.}
  \label{fig:allmerge1}
\end{figure*}

\begin{figure*}
  \centering
  \includegraphics[width=\textwidth]{allmerge2.pdf}
  \caption{A continuation of Figure~\ref{fig:allmerge0}.}
  \label{fig:allmerge2}
\end{figure*}

\begin{figure*}
  \centering
  \includegraphics[width=\textwidth]{allmerge3.pdf}
  \caption{A continuation of Figure~\ref{fig:allmerge0}.}
  \label{fig:allmerge3}
\end{figure*}

\begin{figure*}
  \centering
  \includegraphics[width=\textwidth]{allmerge4.pdf}
  \caption{A continuation of Figure~\ref{fig:allmerge0}.}
  \label{fig:allmerge4}
\end{figure*}

\begin{figure*}
  \centering
  \includegraphics[width=\textwidth]{allmerge5.pdf}
  \caption{A continuation of Figure~\ref{fig:allmerge0}.}
  \label{fig:allmerge5}
\end{figure*}

\begin{figure*}
  \centering
  \includegraphics[width=\textwidth]{allmerge6.pdf}
  \caption{A continuation of Figure~\ref{fig:allmerge0}.}
  \label{fig:allmerge6}
\end{figure*}

\begin{figure*}
  \centering
  \includegraphics[width=\textwidth]{allmerge7.pdf}
  \caption{A continuation of Figure~\ref{fig:allmerge0}.}
  \label{fig:allmerge7}
\end{figure*}

\begin{figure*}
  \centering
  \includegraphics[width=\textwidth]{allmerge8.pdf}
  \caption{A continuation of Figure~\ref{fig:allmerge0}.}
  \label{fig:allmerge8}
\end{figure*}

\section{Cause of Suppressed Star Formation}\label{app:cause_qui}

\begin{figure*}
  \centering
  \includegraphics[width=\textwidth]{SFR_Mgas_tdep.pdf}
  \caption{\textbf{The suppression of star formation in the bimodal simulation is associated with both a reduction in gas mass as well as an increase in the depletion time.} The drop in star formation (blue line) at $\sim2.5-3\,\Gyr$ is associated with both a reduction in the total gas mass (red line) as well as an increase in the depletion time (green line). This shows that the SFR suppression is a result of both less gas mass and more inefficient star formation.}
  \label{fig:SFR_Mgas_tdep}
\end{figure*}

\begin{figure}
  \centering
  \includegraphics[width=242.26653pt]{MdotBH_rsep.pdf}
  \caption{\textbf{The bimodal merger is associated with high accretion rates onto the central SMBH.} Around the time of each pericentric passage, and slightly after, the accretion rate (blue line) becomes very high. Here, it is shown as a fraction of the Eddington accretion rate. During the merger, the accretion rate reaches as high as Eddington at some times, but is always above $10\%$. The orbital separation between the central and satellite is shown in the orange line.}
  \label{fig:MdotBH_rsep}
\end{figure}

\end{document}
