\documentclass[linenumbers, twocolumn]{aastex631}

\usepackage{graphicx}
\usepackage{amsmath}
\usepackage{amssymb}
\usepackage{newtxtext,newtxmath}
\usepackage{hyperref}
\usepackage{gensymb}

\newcommand{\vdag}{(v)^\dagger}
\newcommand\aastex{AAS\TeX}
\newcommand\latex{La\TeX}

\newcommand{\Msun}{\ensuremath{M_{\odot}}}
\newcommand{\Gyr}{\ensuremath{\textrm{Gyr}}}
\newcommand{\Myr}{\ensuremath{\textrm{Myr}}}
\newcommand{\yr}{\ensuremath{\textrm{yr}}}
\newcommand{\kpc}{\ensuremath{\textrm{kpc}}}
\newcommand{\pc}{\ensuremath{\textrm{pc}}}
\newcommand{\tocite}{\textcolor{blue}{cite}}
\newcommand{\FeH}{\ensuremath{[\textrm{Fe}/\textrm{H}]}}
\newcommand{\MgFe}{\ensuremath{[\textrm{Mg}/\textrm{Fe}]}}
\newcommand{\alphaFe}{\ensuremath{[\alpha/\textrm{Fe}]}}
\newcommand{\tform}{\ensuremath{t_{\textrm{form}}}}
\newcommand{\dex}{\ensuremath{\textrm{dex}}}

\newcommand{\red}[1]{\textcolor{red}{#1}}

% \received{March 1, 2021}
% \revised{April 1, 2021}
%\accepted{\today}

\shorttitle{The Milky Way's Phoenix Phase}
\shortauthors{Beane et al.}

\graphicspath{{./}{fig/}}

\begin{document}

\title{Rising from the Ashes: How the Milky Way Got Its Scars}

\author{Angus Beane}
\affiliation{Center for Astrophysics $|$ Harvard \& Smithsonian, Cambridge, MA, USA}

\author{Lars Hernquist}
\affiliation{Center for Astrophysics $|$ Harvard \& Smithsonian, Cambridge, MA, USA}

\author{et al}

\begin{abstract}
  The nuclear abundance distribution of stars encodes the history of the gas-phase abundance in the Milky Way. Without a large, unbiased sample of stellar ages, the exact timing and nature of this history must be \textit{inferred} from the abundances. In the two-dimensional plane of \alphaFe-\FeH, it is now clear that two separate populations exist - the low-$\alpha$ and high-$\alpha$ sequences. A restatement of this fact is that the distribution of \alphaFe{} is bimodal at a fixed Fe. Structure in the nuclear abundance distribution can arise from many processes - proposals include radial migration, high-redshift clump formation, and various effects associated with galaxy mergers, among others. In this work, we demonstrate another possible avenue for structure formation with clear observational predictions. We propose that the Galaxy underwent a starburst followed by a brief (hundreds of Myr) quiescent phase. A natural consequence of this quiescent phase is that stars in the valley of the bimodality do not form because: (1) the absence of enrichment from high-mass stars leads to a rapid reduction in \alphaFe{}, and (2) any time the gas spends in the abundance valley is deemphasized in the present day distribution because the star formation rate is lower. With a set of idealized merger simulations, we demonstrate the feasibility of this proposal. The starburst and quiescent phase can result from a merger, but it is not a necessary component of the theory. This ``phoenix hypothesis'' predicts a gap in stellar ages at a fixed \FeH{} and that stars which form directly after this gap would have lower \alphaFe{} than stars which form slightly later.
\end{abstract}

\keywords{Classical Novae (251) --- Ultraviolet astronomy(1736) --- History of 
astronomy(1868) --- Interdisciplinary astronomy(804)}

\section{Introduction} \label{sec:intro}
Elements heavier than hydrogen are produced through nuclear fusion. After Big Bang Nucleosynthesis, this only occurs in compact objects. The distribution of elemental abundances in the gas-phase of a galaxy is determined by a complicated combination of physical processes - stellar evolution and supernovae, gas accretion, galaxy mergers, gas outflows from stellar and AGN feedback, metal mixing and diffusion, etc.

By necessity, stars inherit the constituitive properties of the gas cloud from which they formed. Moreover, the surface abundance of stars do not change over time.\footnote{For the most part.} We therefore have the unique opportunity to examine the historical record of the gas-phase abundance of a galaxy by way of the distribution of the surface abundances of stars. Because the processes which give rise to this distribution are complex, there is almost certainly some structure in this distribution for every galaxy. However, it has only been definitively measured in the Milky Way.

The distribution of elemental abundances is a high dimensional space (\red{e.g. up to 8 billion elements by XYZ}). However, this space is highly degenerate, and so the effective number of dimensions is much smaller - even possibly compressed to just \FeH and age \citep{2019ApJ...883..177N}.

Two elements in particular have received particular interest - iron and $\alpha$-elements (elements produced through the $\alpha$-process, typically tracked with just Mg). Type Ia and Type II supernovae are the main contributors of elemental enrichment. Iron is broadly produced in both types, and so its abundance is a proxy for the total metallicity of a star. $\alpha$-elements, on the other hand, are mainly produced in Type II supernovae. The ratio of $\alpha$-elements to iron is then a measure of the relative contributions of Type Ia and II to the enrichment of a parcel of gas. It has therefore become common to compress the high-dimensional abundance space to the two dimensional \FeH-\alphaFe plane.

In this plane, it is now well-established that a significant bimodality exists in the Milky Way \red{cite a boatload}.





Elemental enrichment is accounted for in modern galaxy formation models, though the validity of the yield tables is suspect. It is not known if the yields (i.e., the amount of metals produced by the SN) used in such models are accurate. This issue is further compounded by the fact that the mixing of metals is unresolved and is sensitive to numerical choices (e.g., Eulerian codes being generally more diffusive than Lagrangian). To make matters even worse, the inflows and outflows of gas from galaxies is sensitive to the choices of the feedback model, which has a strong impact on a galaxy's elemental history.




The last\footnote{i.e., not ongoing} significant merger was between the
proto-Milky Way disk and the {\em Gaia}-Sausage-Enceladus (GSE)
(\textcolor{red}{spell check}) satellite galaxy. The stellar debris from this
merger constitutes $\sim50\%$ \textcolor{red}{check} of the inner ($r<?\,\kpc$)
stellar halo. It may have also led to a tilted, triaxial dark matter halo
(\textcolor{red}{cite}).

The GSE merger is often invoked (\textcolor{red}{cite}) to explain the observed
bimodality in the \textcolor{red}{alpha iron} abundance plane
(\textcolor{red}{cite}). Because GSE-mass galaxies are expected to have a lower
star formation efficiency than proto-Milky Way-mass galaxies
(\textcolor{red}{cite}), the gas from GSE should be relatively metal poor. The
gas from GSE thus dilutes the gas in the proto-Milky Way, ``resetting'' the
chemical abundance of the Milky Way.

However, it has also been claimed that the \textcolor{red}{alpha iron}
bimodality can be explained through secular processes. \textcolor{red}{Sentence
explaining the basic concept.} The argument is not that the GSE merger did not
happen, but rather that it is not {\it necessary} and might not even be {\it
sufficient} to explain the bimodality. Ongoing work to detect the bimodality in
external galaxies may shed further light on the topic (\textcolor{red}{cite}).

Investigating GSE-like mergers in cosmological simulations is a rich and active
area of research. Many mergers believed to be similar to the GSE merger have
been identified in the literature (\textcolor{red}{cite}). \textcolor{red}{Blah
blah argue blah blah. And also blah blah argues blah blah.}

While much has been learned from GSE-like mergers in cosmological simulations,
they are not conducive to controlled experiments. It is possible to change the
mass of satellites through genetic modification (\textcolor{red}{Pontzen}), and
the GSE analog can be removed entirely (\textcolor{red}{Cooke cite}). However,
to our knowledge, no method has been applied to change the orbital parameters of
such a merger. It is also somehwat unclear if the circumgalactic media of
proto-Milky Way mass galaxies at $z\sim2$ are properly simulated.

In this work, we present hydrodynamic simulations of a controlled merger between
a GSE analog and the $z\sim2$ proto-Milky Way. 

\begin{figure*}
  \centering
  \includegraphics[width=\textwidth]{figure1.pdf}
  \caption{\textbf{The abundance bimodality seen in the Milky Way can be reproduced in some idealized merger simulations.} In the upper panels, we show the distribution of stars in the \MgFe-\FeH plane. The left panel shows the observed distribution in the Milky Way from ASPCAP \red{cite}, while the right two panels show two idealized merger simulations. The idealized merger simulations are nearly identical, except that the bimodal simulation has a starting radius of $129\,\kpc$, while the unimodal simulation has a starting radius of $142\,\kpc$. We emphasize that the labels ``unimodal'' and ``bimodal'' are of the \textit{outcome} of the simulation, and do not reflect a particular choice in the setup. The bottom panels show the distribution of \MgFe at fixed \FeH. The colors indicate the fixed \FeH values, which are $-0.5$, $-0.375$, $-0.25$, and $0.25$. The Milky Way (left panels) exhibits a strong bimodal distribution of \MgFe at various \FeH. The idealized merger simulation marked as bimodal (center panels) also exhibits a bimodal distribution of \MgFe, though the structure is not as strongly defined. The idealized merger simulation marked as unimodal exhibits only weak structure, if any at all.}
  \label{fig:fig1}
\end{figure*}

\section{Methods}\label{sec:methods}
\subsection{Simulation Setup}\label{ssec:setup}
Our goal in setting up the simulations in this work is not an attempt at matching any Milky Way observable in any detail. Our aim is to construct a setup resembling the merger between the Milky Way and GSE at $z\sim2$. However, there are great uncertainities associated with the state of the Milky Way and of GSE at this time. In Appendix~\ref{app:ics}, we discuss in more detail the choices made in tuning our setup to arrive at a system which is reasonably useful for understanding the Galaxy.

Here, we provide an overview of some salient details of our setup. The interested reader is referred to Appendix~\ref{app:ics} for more details. In isolation, each galaxy is a compound halo setup, with a Hernquist dark matter halo and a gaseous halo with a $\beta$-profile. The dark matter halo is initialized to be in gravitational equilibrium with the total potential. The gaseous halo is in gravito-hydrostatic equilibrium, where the temperature is allowed to vary as a function of radius. The azimuthal velocity of the gaseous halo is given as a fraction of the circular velocity. There is no initial stellar disk or bulge, and the gaseous halo is initially metal-free. All star particles and metals are formed self-consistently. Both the central and satellite galaxies are setup in this way.

In order to combine the galaxies, we follow \citet{2021ApJ...923...92N}, and place the satellite galaxy on a retrograde orbit. In the fiducial simulation, the satellite is placed at the virial radius with the virial velocity with a circularity of $0.5$. In order to test minor changes to the orbit, we ran a grid of simulations with $\pm10\%$ in each the starting radius and velocity, and $\pm0.1$ in the circularity, for a total of $27$ simulations.


\section{Results}\label{sec:results}
\subsection{Abundance Distribution}
In Figure~\ref{fig:fig1}, briefly discussed in the Introduction, we show the abundance distribution of the Milky Way as well as two of our idealized merger simulations. A number of our idealized simulations exhibit either a bimodal or unimodal abundance distribution, and so we have selected two which are representative examples of 

\begin{figure*}
  \centering
  \includegraphics[width=\textwidth]{before_after.pdf}
  \caption{\textbf{The high-$\alpha$ sequence forms before the merger, the low-$\alpha$ sequence forms after the merger.} This plot shows the sequence of events leading to the build-up of the low- and high-$\alpha$ sequences for our fiducial bimodal simulation. We have separated the high- and low-$\alpha$ sequences by a dashed line at $-0.1\FeH + 0.31$, which was chosen by eye to lie in the trough. The left panel shows all star particles in our solar neighborhood cut. The middle left panel shows the star particles that form before the merger ($\tform < 1.5\,\Gyr$), which form a weak sequence of star particles at the lowest \FeH and highest \MgFe. The middle right panel shows the star particles that form during the merger ($1.5\,\Gyr < \tform < 2.5\,\Gyr$). These star particless form the portion of the high-$\alpha$ sequence closest to the trough, and the density of star particless is higher than those that form before. The middle right panel shows the star particles which form after the merger ($\tform > 2.5\,\Gyr$). These star particles form almost entirely below the trough.}
  \label{fig:before_after}
\end{figure*}

\begin{figure}
  \centering
  \includegraphics[width=\columnwidth]{before_after_sfh.pdf}
  \caption{\textbf{At fixed metallicity, the low- and high-$\alpha$ sequence are cleanly separated in time, with an intervening quiescent period.} Here we show the star formation history of the high- and low-$\alpha$ sequences. The blue (low-$\alpha$) and orange (high-$\alpha$) histograms correspond to the dashed line cut made in the \MgFe-\FeH plane in Figure~\ref{fig:before_after}, at a fixed metallicity of $\FeH\sim0$. One can see that there is a nearly perfect separation in time between the formation of the high-$\alpha$ and low-$\alpha$ sequence, except for a brief overlap which can be attributed to the inadequacy of the linear cut model of the two sequences. Separating the formation periods lies a quiescent period with a duration of $\sim300\,\Myr$. \red{make sfr per dex}.}
  \label{fig:before_after_sfh}
\end{figure}

\begin{figure}
  \centering
  \includegraphics[width=\columnwidth]{before_after_sfh_by_iron.pdf}
  \caption{\textbf{The timing of the quiescent period which divides the high- and low-$\alpha$ sequences is metallicity dependent.} The star formation history of stars at various fixed metallicities ($\FeH\sim0$, $-0.25$, and $-0.5\,\textrm{dex}$). As in Figure~\ref{fig:before_after_sfh}, there is a quiescent period period which separates the formation of the low- and high-$\alpha$ sequence \red{maybe a better way to show this?}. However, one can see that the timing of the quiescent period is metallicity-dependent. It occurs earlier at lower metallicity, by $\sim300\,\Myr$ for the metallicities shown in this plot. \red{make sfr per dex}.}
  \label{fig:before_after_sfh_by_iron}
\end{figure}

\begin{figure*}
  \centering
  \includegraphics[width=\textwidth]{SFR_alpha.pdf}
  \caption{\textbf{A global suppression of star formation is associcated with a decrease in \MgFe{}, which is seen in the bimodal simulation but not in the unimodal simulation.} Here, we show both the SFR of the central galaxy ($r<15\,\kpc$) and the median \MgFe{} for gas at $2\,\kpc<r<5\,\kpc$ at a fixed \FeH{} bin centered on $-0.2$ with width $0.02\,\dex$. The left panel shows the bimodal simulation while the right panel shows the unimodal simulation. The time of the merger (i.e., the second pericenter) is indicated by the vertical dashed line. In the bimodal simulation, the star formation is suppressed after the merger. This suppression of star formation is associated with a sudden drop in the median \MgFe{} of the gas. Neither the suppression of star formation nor the drop in \MgFe{} are seen in the unimodal simulation.}
  \label{fig:SFR_alpha}
\end{figure*}


\input{discussion.tex}

\input{conclusions.tex}

\bibliography{ref}{}
\bibliographystyle{aasjournal}

% \input{appendix.tex}

\end{document}
