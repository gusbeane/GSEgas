\section{Conclusion}\label{sec:conclusion}
The \alphaFe{}-\FeH{} plane of stellar abundances is a record of the gas-phase abundances of the Galaxy. In this plane, a significant bimodality has now been definitively measured. Proposals for its formation include radial migration, particular gas infall scenarios, and galaxy mergers.

In this work, using idealized merger simulations, we have shown that (1) a starburst, followed by (2) a brief ($\sim300\,\Myr$) quiescent period, followed by (3) rejuvenation can lead to a bimodal abundance structure. We found that the starburst-to-quiescent period allowed for prodigious Fe-enrichment without any corresponding $\alpha$-enrichment. This ``phoenix hypothesis'' would predict that, for stars at a given \FeH{}, there would be a $\sim300\,\Myr$ gap in stellar ages at $\sim8\,\Gyr$. Furthermore, stars which form directly after this gap would have lower \alphaFe{} than stars which form slightly later ($\sim1\,\Gyr$).

The bimodality in the \alphaFe{}-\FeH{} plane of stellar abundances points to deeper, unobserved structure. That structure is a record of the complicated processes that gave rise to the Galaxy. In the absence of highly precise stellar ages, that history must be largely inferred.