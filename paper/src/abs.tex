\begin{abstract}
    The nuclear abundance distribution of stars encodes the history of the gas-phase abundance in the Milky Way. Without a large, unbiased sample of highly precise stellar ages, the exact timing and nature of this history must be \textit{inferred} from the abundances. In the two-dimensional plane of \alphaFe-\FeH, it is now clear that two separate populations exist -- the low-$\alpha$ and high-$\alpha$ sequences. Structure in the nuclear abundance distribution can arise from many processes -- proposals include specific gas infall scenarios, radial migration, high-redshift clump formation, and various effects associated with galaxy mergers, among others. In this work, we demonstrate another possible avenue for structure formation with clear observational predictions. We propose that the Galaxy underwent a starburst followed by a brief (hundreds of Myr) quiescent phase -- i.e., the Galaxy underwent a post-starburst rejuvenation sequence at $z\sim2$. A natural consequence of the quiescent phase is that stars in the valley of the bimodality do not form because: (1) the absence of enrichment from high-mass stars leads to a rapid reduction in \alphaFe{}, and (2) any time the gas spends in the abundance valley is deemphasized in the present day distribution because the star formation rate is lower. With a set of idealized merger simulations, we demonstrate the feasibility of this proposal. This ``phoenix hypothesis'' predicts a $\sim300\,\Myr$ gap in stellar ages at a fixed \FeH{} and that stars which form directly after this gap would have lower \alphaFe{} than stars which form slightly ($\sim1\,\Gyr$) later.
  \end{abstract}
  
\keywords{Classical Novae (251) --- Ultraviolet astronomy(1736) --- History of astronomy(1868) --- Interdisciplinary astronomy(804)}
