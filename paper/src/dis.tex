\section{Discussion}\label{sec:discussion}
We have demonstrated that a bimodal structure in the elemental abundance plane is associated with a brief quiescent phase in our idealized merger simulations. We now construct a possible framework for understanding the formation of the bimodality, discuss its relation to prior work, and then discuss predictions and connections to observations.

\subsection{Quiescence Leads to Bimodality}\label{ssec:formqui}
We executed a series of idealized merger simulations in which we modified the starting radius and velocity by $\pm10\%$ and the circularity by $\pm0.1$, for a total of $27$ simulations. The central and satellite galaxies, which are meant to resemble the Milky Way and GSE at $z\sim2$, are otherwise identical across the simulations. Some of these simulations induce a bimodality, while others do not. We have examined a representative of each scenario in detail.

We described a series of results in Section~\ref{sec:results} which give a natural explanation for the separation between the high- and low-$\alpha$ sequences in our simulations. First, a merger induces a starburst in the simulation. If this starburst does not induce a quiescent period, then the median \MgFe{} of the gas briefly increases before continuing along the typical evolutionary path (right panel of Figure~\ref{fig:SFR_alpha}). However, if there is a quiescent period, the median \MgFe{} drops for the duration of the quiescent period before recovering to the typical path (left panel of Figure~\ref{fig:SFR_alpha}).

This drop can be understood in terms of the mechanics of the galaxy formation model as described in Section~\ref{ssec:gfm}. When a star particle forms, it only takes $\sim40\,\Myr$ before the last Type II SN has gone off and its material enrichment is dominated by Fe from Type Ia SNe. As a result, during the quiescent phase which lasts a few hundred Myr, there is relatively little $\alpha$-element production. On the other hand, star particles are still able to generate Fe through Type Ia SNe which have a longer period of influence. The fact that the quiescent period is preceded by a starburst is important. Even though star particles produce Fe through the Type Ia channel for Gyr, younger star particles still produce much more Fe than older star particles. In the TNG model, the number of Type Ia SNe decays with time as $(t/\tau_8)^{-s}$, where $\tau_8=40\,\Myr$ is the lifetime of an $8\,\Msun$ star and $s=1.12$. Therefore, the burst of star formation at $t\sim2.25\,\Gyr$ is able to generously produce Fe starting at $2.5\,\Gyr$, when is indeed when the drop in \MgFe{} is the most steep.

\subsection{Connection to Cosmological Simulations}\label{ssec:cosmo}
As discussed in Section~\ref{sec:intro}, several authors have examined the formation of abundance plane structure in cosmological simulations. Of most interest to us is the zoom Au~23 in \citet{2018MNRAS.474.3629G}. This galaxy, one of six considered in their work, exhibits a bimodality that extends beyond the inner disk. The interpretation given by the authors is of a ``shrinking'' gaseous disk. This is equivalent to saying that the outer disk becomes depleted of gas. This shrinking of the disk, which occurs at $t_{\textrm{lookback}}\sim6\,\Gyr$, is associated with a dip in the SFR at that radius and a decrement in the median \alphaFe{} of $\sim0.05\,\dex$ (their Figure~2), which shortly after recovers. This sequence of events is more extended than in our work, but it strongly resembles the scenario in Figure~\ref{fig:before_after_sfh}.

\citet{2018MNRAS.477.5072M} found that Milky Way-like bimodalities are rare in EAGLE, occuring in $\sim5\%$ of galaxies. \citet{2021MNRAS.501..236D,2022MNRAS.515.1430D} showed that merger-induced quenching in zooms can occur in the EAGLE model \citep[see also][]{2017MNRAS.465..547P}. However, the situation may be different in the lower resolution large box. Furthermore, if the proposed starburst-quenching phase is driven by AGN feedback, then the outcome of any particular cosmological simulation with regards to the bimodality is intimately tied to its AGN model. Unfortunately, such models are highly uncertain \citep[e.g.][]{2022MNRAS.511.3751H}.

\citet{2023arXiv231016085K} explored the impact of quenching in the Magneticum Pathfinder suite. They found that galaxies which quench undergo a starburst followed by an AGN-driven quenching phase. In the post-starburst regime, they claim galaxies are $\alpha$-enhanced. We do find that the bulk stellar \MgFe{} is enhanced after the merger in our bimodal simulation compared to the isolated simulation, but only at the $\sim0.01\,\dex$ level.

\subsection{Infall Interpretations}\label{ssec:dilute}
In some previous work, it was reported that the bimodality is a consequence of a sudden deposition of metal-poor, $\alpha$-poor gas by a satellite or cosmological filament -- i.e., a ``dilution'' \citep{2020MNRAS.491.5435B,2021MNRAS.503.5846R}. The separation of the sequences follows from the rapidity of the dilution. This was elaborated upon by \citet{2021MNRAS.503.5868R} who described a zoom where the low-$\alpha$ disk forms out of a relatively pristine cosmological filament. This disk is inclined relative to the high-$\alpha$ disk, with the two disks later tidally realigning. The longer-standing two-infall class of models argue that the two sequences diverge due to two episodic accretion episodes, with some possible enrichment of the second episode arising from an associated satellite \citep{1997ApJ...477..765C,2009IAUS..254..191C,2017MNRAS.472.3637G,2019A&A...623A..60S}.

We have shown that minor changes to the orbit of our idealized merger can result in outcomes that are either bimodal or unimodal. The content of gas that is delivered to the system is nearly identical regardless of the orbit, and so the dilution interpretation is not applicable to our simulations. That being said, a removal of gas from the system either through star formation or through ejection could make dilution from infalling gas more efficient, so the two scenarios are not mutually exclusive.

It was recently elaborated by \citet{2024arXiv240511025S} that these models also argue for a star formation gap between the two accretion episodes \citep[see also][]{1996ASPC...92..307G,1998A&A...338..161F,2000A&A...358..671G,2015A&A...578A..87S,2020A&A...640A..81N}. This gap is starvation-driven and can last several \Gyr{}. The present work argues for a starburst-driven quiescence followed by a rapid rejuvenation, with the entire process taking less than $1\,\Gyr$ and the gap only lasting a few hundred \Myr{}. The physical origin and some details are different, but one can appreciate that the bimodality arises from the same physical process.\footnote{Compare Section~\ref{ssec:formqui} to the first key result in the Conclusions of \citet{2024arXiv240511025S}.}

\subsection{Observational Connections: High Redshift Quenching}\label{ssec:obshiz}
If the Milky Way underwent a starburst to quiescent to rejuvenation sequence, as we argue in this paper, then an an analogue at $z\sim2$ would be identified as a post-starburst galaxy (PSB). With abundance matching, we expect the Milky Way's total stellar mass to be $\sim10^{10.3}\,\Msun$ at $z\sim2$ \citep{2013ApJ...771L..35V}. A number of authors have explored PSBs and quiescent galaxies at slightly higher masses at $z\sim2$, with large advances in the post-JWST era.

First, PSBs are not uncommon. \citet{2023ApJ...953..119P} found that in massive galaxies ($M_* > 10^{10.6}\,\Msun$) the fraction of PSBs (inferred ages $< 800\,\Myr$) increases from $\sim2.7\%$ ($99/3655$) at $1.0 < z < 1.44$ to $\sim8\%$ ($89/1118$) at $2.16 < z < 2.5$ \citep[see also][]{2012ApJ...745..179W,2019ApJ...874...17B}. Later, \citet{2024arXiv240417945P} found that $\sim10\%$ of galaxies at $\sim10^{10.3}\,\Msun$ are quenched \citep[consistent with][]{2013ApJ...777...18M}, and $\sim30\%$ of their quiescent sample is a PSB at $z\sim2$. If these galaxies can be quickly rejuvenated, as the system studied in this work would suggest, then the total fraction of galaxies that go through a starburst-quenching phase may be higher.

Furthermore, \citet{2023arXiv231215012C} found that lower mass quiescent galaxies (towards $10^{10.3}\,\Msun$) tend to be younger and more disky, pointing to a merger driven scenario. \citet{2023arXiv231212207A} found that at even lower masses ($<10^{9.5}\,\Msun$) There is also evidence that the AGN quenching mechanism, which we believe to be responsible for the quenching in our system (Appendix~\ref{app:cause_qui}), is operating at these redshifts \citep[e.g.][and references therein]{2023arXiv230806317D,2024arXiv240417945P,2024arXiv240518685M,2024Natur.630...54B}.

Overall, the starburst-driven quenching mechanism proposed in this work is allowed by high-redshift observations.

\subsection{Direct Search for Quiescence}\label{ssec:obsqui}
Figure~\ref{fig:before_after_sfh_by_iron} indicates a very direct observational test of the mechanism proposed in this work: for disk stars at a given \FeH{}, there should be a gap of $\sim300\,\Myr$ in ages at $\sim8\,\Gyr$, though in the Milky Way the gap could be larger. With a survey of properly chosen stars, this gap could be directly measured with a modestly sized (few hundred) sample of old stars with age uncertainties of $\sim1\%$. To our knowledge, the best method at these ages is differential analysis of solar twins, which can provide an age uncertainty of $\sim5\%$ \citep[e.g.][]{2014ApJ...795...23B,2018MNRAS.474.2580S}. However, this has only been applied to stars with solar metallicity, where there is not a clear separation between the high- and low-$\alpha$ sequences (though a gap in ages may still be present). A sort of differential approach could be applied also at lower metallicities, which would just lack the absolute age calibration that the Sun provides.

The gap could also be indirectly probed by a larger sample of slightly less precise ages. Astroseismology appears to be the most promising avenue. Currently the largest sample is APOKASC2 \citep{2018ApJS..239...32P}, which has $\sim3400$ stars with ages measured to $<20\%$ precision.\footnote{Errors here taken to be the maximum of the upper and lower age estimates in the public APOKASC2 catalog.} The upcoming PLATO mission is looking to measure $\sim20\textrm{k}$ stars with ages measured to $<10\%$ precision \citep{2024arXiv240605447R}. Further work is needed to determine how well these surveys could constrain the gap.

In addition to the presence of the gap, our work also predicts that: (1) the timing of the gap is \FeH{} dependent, and that (2) stars which form immediately after the gap are $\alpha$-poor compared to stars which form slightly later (see Figure~\ref{fig:alpha_vs_tform}). However, both of these tests rely on the precise age estimates.

\subsection{Future Work}
A natural next step would be to extend the idealized simulations in this work to a wider range of orbits, galaxy properties, and feedback models. However, the unrealism of our setup, which aids interpretation, limits its applicability to the real universe. And so something along the lines of the genetic modification technique to explore various mergers as done in this work may be useful \citep{2016MNRAS.455..974R,2017MNRAS.465..547P}.

There is, of course, still great uncertainties in the stellar evolution models commonly adopted by different groups. Initial work on systematically exploring the stellar evolution parameters has been done by \citet{2017A&A...605A..59R,2021MNRAS.508.3365B}. Exploring these variations in the simulations presented in this work would be interesting, though exploring their interactions with brief quiescent periods in simpler chemical evolution models may be a better first step.

There is also the perennial problem of diffusion within the hydrodynamics solver. In purely Lagrangian solvers, there is no diffusion between resolution elements, while in Eulerian codes the diffusion can be quite high.\footnote{No galaxy formation simulation is fully Lagrangian since there must be, at a minimum, mass exchange between star particles and gas. Here we just mean that there is no mass exchange between gas elements.} AREPO limits the numerical diffusion by allowing the mesh to move in a quasi-Lagrangian manner, and using a second-order solver \citep{2010MNRAS.401..791S}. In FIRE-2 \citep{2018MNRAS.480..800H}, which uses the Lagrangian code GIZMO \citep{2015MNRAS.450...53H}, a subgrid turbulent metal diffusion model was used. It would be interesting to see how models with different diffusivity properties would relax or strengthen the necessity of a quiescent period to produce a bimodality.

% JWST quiescent galaxies https://arxiv.org/abs/2404.12432
